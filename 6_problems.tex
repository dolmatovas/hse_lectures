\documentclass[12pt]{article}

\usepackage{preamble_problemset}
\usepackage{amsmath}

\begin{document}
\noindent Количественные финансы, осень 2025\hfill Семинар 4\\
\today

\hrulefill


\begin{problem}
    Найти стоимость вечного американского пут-опциона:
    $$
        P(s) = \sup_{\tau \in \mathcal{T}} \E e^{-r\tau} (K-S(\tau))
    $$где супремум берётся по всем марковским моментам $\mathcal{T}$.
\end{problem}

\textit{Решение}

    \begin{itemize}

        \item Задача однородная по времени. Ищем решение в классе моментов остановки:
`        $$
            \tau_L = \inf\{t \geq 0, S_t = L\}
        $$
        \item Исполняем опцион в первый момент, когда цена пробъёт уровень $L < s$.
        \item Найдем ожидаемую выплату для такой стратегии:
        $$
            V_L(s) = \E e^{-r\tau}(K-S_{\tau})
            = (K-L)\E e^{-r\tau}
        $$Считаем, что $e^{-r\tau}(K-S_{\tau})=0$ при $\tau = \infty$ (выплаты не происходит).
        \item Нужно найти преобразование Лапласа от $\tau_L, \phi(r) = \E e^{-r\tau}$.
        \item При $r > 0$:
        $$
            \phi(r) = \E e^{-r\tau} = \E e^{-r\tau} \mathbb{I}(\tau < \infty)
        $$
        \item БД со сносом $X_t = \mu t + W_t$
        \item Момент остановки $\tau = \inf\{t \geq 0: X_t = a\}$. 
        \item Введём также $\tau_n = \inf\{t \geq 0: X_t = a \lor X_t = -n\}$
        \item $\tau_n \to \tau$ при $n\to\infty$
        \item Найдем $\sigma$ такое, что $M_t$ -- мартингал:
        $$
            M_t = e^{\sigma X_t - r t}
        $$
        \item По теореме Дуба:
        $$
            1 = \E M_{\tau_n} = \E e^{\sigma a - r\tau} \mathbb{I}(\tau \leq \tau_n)
            + \E e^{-\sigma n - r\tau_n} \mathbb{I}(\tau > \tau_n)
        $$ При $n\to \infty$ правая часть сходится к:
        $$
            \E e^{\sigma a - r\tau} \mathbb{I}(\tau \leq \tau_n) = 
            e^{\sigma a} \E e^{-r\tau}
        $$откуда
        $$
            \E e^{-r\tau} = e^{-\sigma a}
        $$
        \item $\sigma = -\mu + \sqrt{\mu^2 + 2r}$, откуда:
        $$
        \E e^{-r\tau} = e^{-(-\mu + \sqrt{\mu^2 + 2r}) a}
        $$
    \end{itemize}



Барьерный опцион

$$
    u_t = \mu u_x + 0.5 \sigma^2 u_{xx}
$$

Хотим избавиться от $u_x$. Ищем решение в виде $u(t, x) = e^{\alplha x} v(t, x)$

\begin{align*}
    &u_t = e^{\alpha x} v_t\\
    &u_x = \alpha u + e^{\alpha x} v_x = e^{\alpha x}( \alpha v + v_x)\\
    &u_{xx} = \alpha u_x + \alpha e^{\alpha x} v_x + e^{\alpha x} v_{xx}\\
    &u_{xx} = e^{\alpha x} \left( \alpha^2 v + 2 \alpha v_x + v_{xx}\right)\\
\end{align*}
Подставляем в уравнение:
\begin{align*}
    &v_t = \mu \alpha v + \mu v_x + 0.5 \sigma^2 \alpha^2 v + \sigma^2 \alpha v_x + 0.5 \sigma^2 v_{xx} =\\
    &= v (\mu \alpha + 0.5 \sigma^2 \alpha^2) + v_x (\mu + \sigma^2 \alpha )
    + 0.5 \sigma^2 v_{xx}
\end{align*}Положим $\alpha = -\dfrac{\mu}{\sigma^2}$, получим уравнение:
$$
    v_t = 0.5 \sigma^2 v_{xx} - \dfrac{\mu^2}{\sigma^2} v
$$

Пусть $v(t, x)$ -- решение. Тогда $v(t, 2a - x)$ -- тоже решение.  Введём
$w(t, x) = v(t, x) - v(t, 2a - x)$, тогда $w(t, x)$ -- решение, причём $w(t, a) = 0$.

Поэтому 
$$
    u(t, x) = e^{\alpha x} v(t, x) - e^{\alpha (2a - x)} v(t, 2a - x)
$$является решением исходного уравнения с условием $u(t, a) = 0$.

\end{document}

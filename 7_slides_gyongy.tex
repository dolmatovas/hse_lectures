\documentclass[aspectratio=169]{beamer}
\usepackage{preamble_beamer}

\newcommand*{\QEDB}{\null\nobreak\hfill\ensuremath{\square}}

\newtheorem*{question*}{Вопрос}
\newtheorem*{claim*}{Утверждение}

\title[Модели стохастической волатильности]{Лекция 7. Модель Хестона} % The short title appears at the bottom of every slide, the full title is only on the title page

\begin{document}


\begin{frame}{Модель Хестона}
Рассмотрим модель Хестона с нулевой процентной ставкой:
\begin{align*}
\frac{dS_t}{S_t} &= \sqrt{v_t} dW_t, \\
dv_t &= \kappa (\theta - v_t) dt + \xi \sqrt{v_t} dZ_t, \\
dW_t dZ_t &= \rho dt
\end{align*}
\end{frame}

\begin{frame}{Двумерное уравнение Фоккера-Планка}
Для совместной плотности $p(t, S, v)$:
\begin{align*}
\frac{\partial p}{\partial t} = & -\frac{\partial}{\partial v}[\kappa (\theta - v) p] + \frac{1}{2} \frac{\partial^2}{\partial S^2}[S^2 v p] \\
& + \frac{1}{2} \frac{\partial^2}{\partial v^2}[\xi^2 v p] + \rho \xi \frac{\partial^2}{\partial S \partial v}[S v p]
\end{align*}
\end{frame}

\begin{frame}{Интегрирование по $v$}
Интегрируем по $v$ от $0$ до $+\infty$:
\begin{align*}
\int_0^\infty \frac{\partial p}{\partial t} dv = & -\int_0^\infty \frac{\partial}{\partial v}[\kappa (\theta - v) p] dv \\
& + \frac{1}{2} \int_0^\infty \frac{\partial^2}{\partial S^2}[S^2 v p] dv \\
& + \frac{1}{2} \int_0^\infty \frac{\partial^2}{\partial v^2}[\xi^2 v p] dv \\
& + \rho \xi \int_0^\infty \frac{\partial^2}{\partial S \partial v}[S v p] dv
\end{align*}
\end{frame}

\begin{frame}{Анализ слагаемых}
\begin{itemize}
\item[1.] $\int_0^\infty \frac{\partial p}{\partial t} dv = \frac{\partial p}{\partial t}$
\item[2.] $-\int_0^\infty \frac{\partial}{\partial v}[\kappa (\theta - v) p] dv = 0$ (граничные условия)
\item[3.] $\frac{1}{2} \int_0^\infty \frac{\partial^2}{\partial S^2}[S^2 v p] dv = \frac{1}{2} \frac{\partial^2}{\partial S^2} \left[ S^2 \int_0^\infty v p dv \right]$
\item[4.] $\frac{1}{2} \int_0^\infty \frac{\partial^2}{\partial v^2}[\xi^2 v p] dv = 0$ (граничные условия)
\item[5.] $\rho \xi \int_0^\infty \frac{\partial^2}{\partial S \partial v}[S v p] dv = 0$ (граничные условия)
\end{itemize}
\end{frame}

\begin{frame}{Условное математическое ожидание}
Заметим, что:
\[
\int_0^\infty v p(t, S, v) dv = \mathbb{E}[v_t | S_t = S] p(t, S)
\]
где:
\[
\mathbb{E}[v_t | S_t = S] = \frac{1}{p(t, S)} \int_0^\infty v p(t, S, v) dv
\]
\end{frame}

\begin{frame}{Итоговое уравнение}
Получаем уравнение для маргинальной плотности:
\[
\frac{\partial p}{\partial t} = \frac{1}{2} \frac{\partial^2}{\partial S^2} \left( S^2 \mathbb{E}[v_t | S_t = S] p(t, S) \right)
\]

\textbf{Следствие:}
\[
\sigma_{\text{loc}}^2(t, S) = \mathbb{E}[v_t | S_t = S]
\]
\end{frame}

\begin{frame}{Теорема Дьёндь}
\textbf{Теорема (Gyöngy, 1986):}

Для процесса со стохастической волатильностью:
\[
dS_t = \mu_t S_t dt + \sqrt{v_t} S_t dW_t
\]
существует процесс с локальной волатильностью:
\[
dS_t = \mu_t S_t dt + \sigma_{\text{loc}}(t, S_t) S_t dW_t
\]
с одинаковыми одномерными распределениями, где:
\[
\sigma_{\text{loc}}^2(t, S_t) = \mathbb{E}[v_t | S_t]
\]
\end{frame}


\end{document}

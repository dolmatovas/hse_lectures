\documentclass[aspectratio=169]{beamer}
\usepackage{preamble_beamer}

\newcommand*{\QEDB}{\null\nobreak\hfill\ensuremath{\square}}

\newtheorem*{question*}{Вопрос}
\newtheorem*{claim*}{Утверждение}

\title[Фьючерсные и форвардные контракты]{Семинар 7. Фьючерсные и форвардные контракты} % The short title appears at the bottom of every slide, the full title is only on the title page

\begin{document}

\begin{frame}
\titlepage 
\end{frame}

\begin{frame}{Стохастическая процентная ставка}
    \begin{itemize}
        \item Модели со стохастической ставкой:
        \begin{align*}
            &dB_t = r_t B_t dt\\
            &dS_t / S_t = r_t dt + \sigma dW_t
        \end{align*}
        \item Стоимость опциона с пэйоффом $\Phi(S_T)$:
        $$
            \dfrac{V_t}{B_t} = \E^\Q \left[\dfrac{\Phi(S_T)}{B_T} | \F_t\right]
            \Leftrightarrow V_t = \E^\Q \left[e^{-\int_t^T r_u du} \Phi(S_T) | \F_t \right]
        $$
        \item Дисконт-фактор:
        $$
            DF(t, T) = e^{-\int_t^T r_u du}
        $$
        \item Стоимость бескупонной облигации:
        $$
            P(t, T) = \E^Q [DF(t, T) | \F_t]
        $$
    \end{itemize}
\end{frame}

\begin{frame}{Форвардный контракт}
    \begin{block}{Определение}
        Форвардный контракт -- соглашение о покупке актива в момент $T$ по фиксированной цене $K$. Задаётся функцией выплаты $\Phi(S_T) = S_T - K$
    \end{block}

    \begin{block}{Определение}
        Форвардная цена $\mathrm{Forw}(t, T)$-- страйк форвардного контракта, при котором его стоимость равна нулю.
    \end{block}
    \begin{block}{Теорема}
        $$\mathrm{Forw}(t, T) = \dfrac{\E^\Q DF(t, T) S_T}{P(t, T)}
        = \dfrac{S_t}{P(t,T)}$$
    \end{block}
\end{frame}

\begin{frame}{Фьючерсный контракт}
    \begin{block}{Определение}
        Фьючерсный контракт -- соглашение о покупке актива в момент $T$, при котором платежи размазаны по всему сроку. Он обладает следующими свойствами
        \begin{itemize}
            \item На рынке котируется \textbf{фьючерсная цена} $\mathrm{Fut}_t(T)$.
            \item В момент исполнения держатель платит $\mathrm{Fut}_T(T)$ и получает $S_T$.
            \item $\forall t \leq T$ стоимость фьючерсного контракта равна нулю.
            \item В течении интервала времени $(s, t]$ держатель контракта получает платеж $\mathrm{Fut}_t(T) - \mathrm{Fut}_s(T)$
        \end{itemize}
    \end{block}

    \begin{block}{Теорема}
        \begin{itemize}
            \item $\mathrm{Fut}_T(T) = S_T$
            \item $\mathrm{Fut}_t(T)$ -- мартингал относительно риск-нейтральной меры.
            \item $\mathrm{Fut}_t(T) = \E^{\Q} \left[ S_T | \F_t \right]$
        \end{itemize}
    \end{block}
\end{frame}

\begin{frame}{Фьючерсный контракт}
    \begin{block}{Теорема}
        \begin{itemize}
            \item $\mathrm{Fut}_T(T) = S_T$
            \item $\mathrm{Fut}_t(T)$ -- мартингал относительно риск-нейтральной меры.
            \item $\mathrm{Fut}_t(T) = \E^{\Q} \left[ S_T | \F_t \right]$
        \end{itemize}
    \end{block}
        \textit{Доказательство}. Первый пункт очевиден. Докажем второй пункт. Пусть $(x_t, y_t)$ -- реплицирующий портфель (не обязательно самофинансируемый). Тогда:
        \begin{align*}
            &x_t S_t + y_t B_t = 0 \\
            &d\mathrm{Fut}_t(T) = x_t dS_t + y_t dB_t
        \end{align*}
        Из первого уравнения $y_t = -x_t\dfrac{S_t}{B_t}$, откуда
        $$
            d\mathrm{Fut}_t(T) = x_t \left( dS_t - r_t S_t dt \right)
            = x_t S_t \sigma dW_t
        $$\QEDB
    
\end{frame}

\begin{frame}{Форвардная мера}
\only<1>{
    \begin{block}{Определение}
        Форвардная мера $\Q^T$ -- мартингальная мера для numeraire $P(t, T)$. Относительно $\Q^T$ процессы:
        $$
            \overline{B}_t = \dfrac{B_t}{P(t, T)}, \quad \overline{S}_t = \dfrac{S_t}{P(t,T)}
        $$являются мартингалами.
    \end{block}}
\only<2>{
    \begin{itemize}
        \item Относительно $\Q^T$ $\mathrm{Forw}_t(T)$ является мартингалом:
        $$
            \mathrm{Forw}_t(T) = \dfrac{S_t}{P(t,T)} = \overline{S}_t
        $$
        \item Формула ценообразования:
        $$
            \dfrac{V_t}{P(t, T)} = \E^\Q^T \left[ \dfrac{\Phi(S_T)}{P(T, T)} |\F_t \right]
            \to \E^\Q^T \Phi(S_T)
        $$
        \item Формула замены меры:
        $$
            V_t = \E^\Q DF(t,T)\Phi(S_T) = P(t,T) \E^\Q^T \Phi(S_T)
        $$     
    \end{itemize}
}
\end{frame}

\begin{frame}{Форвардные ставки}
    \begin{block}{Определение}Пусть на рынке торгуются бескупонные облигации для всех $T$.
        
        Определим форвардную ставку $f_t(T)$ как:
        $$
            P(t, T) = \exp(-\int_t^T f_t(U) dU)
        $$или 
        $$f_t(T) = -\dfrac{1}{P(t, T)}\dfrac{\partial P(t, T)}{\partial T}$$
    \end{block}

    \textit{Упражнение}. Покажите, что 
    $$
        f_t(T) = \E^{Q_T} [r_T | \F_t]
    $$
\end{frame}

\begin{frame}{Контракты с разными типами расчёта}
    \begin{itemize}
        \item Контракты со спотовым типом расчёта -- спот цены
        \item Контракты с форвардным типом расчёта -- форвардные цены
        \item Контракты с фьючерсным типом расчёта -- фьючерсные цены
    \end{itemize}
\end{frame}

\end{document}

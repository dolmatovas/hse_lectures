\documentclass[12pt]{article}

\usepackage{preamble_problemset}
\usepackage{amsmath}

\begin{document}
\noindent Количественные финансы, осень 2025\hfill Домашнее задание 5\\
\today

\hrulefill

\begin{problem}
    Рассмотрим рынок с тремя активами:
    \begin{align*}
        &dB_t = 0\\
        &dS_t^1 = S_t^1 \sigma_1 dW_t^1 \\
        &dS_t^2 = S_t^2 \sigma_2 dW_t^2
    \end{align*}где $W_t^1, W_t^2$ -- два броуновских движения с корреляцией $\rho$. Найти стоимость обменного опциона с пэйоффом:
    $$
        \Phi(S_T^1, S_T^2) = (S_T^1 - S_T^2)^+
    $$
\end{problem}

\begin{problem}[Формула Дюпира]
    На пятой лекции мы показали, что если определить функцию локальной волатильности по формуле ниже, то цены опционов в модели локальной волатильности будут совпадать с рыночными ценами.
    $$
            \sigma^2_{Dup}(T, K) = \dfrac{\frac{\partial C^M}{\partial T} + r K \frac{\partial C^M}{\partial K}}
            {K^2 \frac{\partial^2 C^M}{\partial K^2}}
    $$Докажите, что для безабритражных рыночных цен колл-опционов $C^M(T, K)$ данное определение корректно, т.е. что величина справа неотрицательная. 
\end{problem}

\begin{problem}[Американский опцион]
    Пусть динамика активов в риск-нейтральной мере задаётся уравнениями (модель Блэка-Шоулза)
    \begin{align*}
        &dB_t = r B_t dt \\
        &dS_t/S_t = r dt + \sigma dW_t
    \end{align*}где $W_t$ -- броуновское движение. 
    Рассмотрим вечный американский колл-опцион $T=\infty$:
    $$
        V(s) = \sup_{\tau \in \mathcal{T}} \E e^{-r\tau} (S_{\tau} - K)^+
    $$где супремум берётся по всем марковским моментам $\mathcal{T}$, $S_0 = s$.
    \noindent
    Рассмотрим стратегию, при которой мы исполняем опцион при достижении цены уровня $L \geq K$:
    $$
        \tau_L = \inf \{t \geq 0 : S_t \geq L\}
    $$Найти стоимость опциона для такой стратегии:
    $$
        V_L(s) = \E e^{-r\tau_L} (S_{\tau_L} - K)^+
    $$При каком $L$ достигается максимум цены?
\end{problem}

\end{document}

\documentclass[aspectratio=169]{beamer}
\usepackage{preamble_beamer}

\newcommand*{\QEDB}{\null\nobreak\hfill\ensuremath{\square}}

\newtheorem*{question*}{Вопрос}
\newtheorem*{claim*}{Утверждение}

\title[Модели стохастической волатильности]{Лекция 7. Модель Хестона} % The short title appears at the bottom of every slide, the full title is only on the title page

\begin{document}

\begin{frame}
\titlepage 
\end{frame}

\begin{frame}{Рекап прошлой лекции}
    \begin{itemize}
        \item Анализ предположений модели Блэка-Шоулза
        \item Вменяемая (implied) волатильность
        \item Формула Бридена-Литценбергера
        \item Модели локальной волатильности
        \item Формула Дюпира
    \end{itemize}
\end{frame}

\begin{frame}{План лекции}
    \begin{itemize}
        \item Модель Хестона: определение
        \item Формула для цены колл-опциона через вероятности исполнения
        \item Прайсинг в моделях и известной хар. функцией
        \item Вычисление хар. функции в модели Хестона
        \item Связь с моделями локальной волатильности. Теорема Дьёнди.
    \end{itemize}
\end{frame}


\begin{frame}{Модель Хестона}
    \begin{block}{Определение}
        Риск-нейтральная динамика в модели Хестона задаётся системой СДУ:
        \begin{align*}
            &\dfrac{dS_t}{S_t} = r dt + \sqrt{v_t} dW_t\\
            &dv_t = \kappa (\theta - v_t) dt + \xi \sqrt{v_t} dZ_t
        \end{align*}
        где $W_t, Z_t$ -- два броуновских движения с корреляцией $\rho$.
    \end{block}

    Процесс для $v_t$ -- процесс с возвратом к среднему:
    $$
        \E v_t = v_0 e^{-\kappa t} + \theta (1 - e^{-\kappa t})
    $$
\end{frame}

\begin{frame}{Формула замены меры в условном мат. ожидании}
        \begin{itemize}
        \item Замена меры:
        $$
            V_t = \E^{\Q} \left[ \left. X \dfrac{B_t}{B_T} \right\vert \F_t \right]
            = \E^{\Q_S} \left[ \left. X \dfrac{S_t}{S_T} \right\vert \F_t \right]
        $$где $\Q_S$ -- мартингальная мера для numeraire $S_t$.
        \item Введём $Y = X \dfrac{S_t}{S_T}$, откуда:
        $$
            \E^{\Q_S} \left[ \left.Y \right\vert \F_t \right]
            = \E^{\Q} \left[ \left.Y \dfrac{S_T}{S_t} \dfrac{B_t}{B_T} \right\vert \F_t \right]
            = \dfrac{e^{-r(T-t)}}{S_t} \E^{\Q} \left[ \left.Y S_T \right\vert \F_t \right]        
        $$
    \end{itemize}
\end{frame}

\begin{frame}{Прайсинг через вероятности исполнения}
    \begin{itemize}
        \item Общая формула для прайсинга:
        \begin{align*}
            &C_t(T, K) = e^{-r(T-t)} \E^\Q \left[ (S_T-K)^+ | \F_t \right] = \\
            &= e^{-r(T-t)} \E^\Q \left[ S_T \mathbb{I}_{S_T > K} | \F_t \right]
            - e^{-r(T-t)} \E^\Q \left[ K \mathbb{I}_{S_T > K} | \F_t \right] = \\
            &= S_t \Q_S(S_T > K | \F_t) - e^{-r(T-t)} K \Q(S_T > K | \F_t) 
        \end{align*}где $\Q_S$ -- мартингальная мера для numeraire $S_t$.
    \end{itemize}
\end{frame}

\begin{frame}{Характеристическая функция}

    \begin{itemize}
        \item Хар. функция как преобразования Фурье:
        $$
            \varphi(u) = \int_{\R} e^{iux} p(x) dx
        $$
        \item Обратное преобразование Фурье:
        $$
            p(x) = \dfrac{1}{2\pi} \int_{\R} e^{-iux} \varphi(u) du
        $$
        \item Функция распределения:
        $$
            F(x) = \int_{-\infty}^x p(y) dy
            = \dfrac{1}{2\pi} \int_{-\infty}^x dy \int_{\R} e^{-iuy} \varphi(u) du
        $$\item Формула обращения:
        $$
            F(x) = \dfrac{1}{2} - \dfrac{1}{\pi} \int_{0}^{\infty} \Re \left[ \dfrac{e^{-iux}\varphi(u)}{iu} du \right] 
        $$
    \end{itemize}
\end{frame}

\begin{frame}{Прайсинг в моделях с известной хар. функцией}
    \begin{itemize}
        \item Пусть $X_t = \log S_t$. Условная хар. функция:
        \begin{align*}
            &\varphi(u) = \E^\Q \left[e^{iuX_T} | \F_t \right] \\
            &\tilde{\varphi}(u) = \E^{\Q_S} \left[e^{iuX_T} | \F_t \right]
        \end{align*}
        \item Связь:
        $$
            \tilde{\varphi}(u) = \dfrac{\varphi(u - i)}{\varphi(-i)}
        $$
        \item \textit{Доказательство}
        \begin{align*}
            \tilde{\varphi}(u) = \E^{\Q_S} \left[ e^{iuX_T} | F_t\right]
            = e^{-r(T-t)}S_t^{-1} \E^{\Q} \left[ S_T e^{iuX_T} | F_t \right] =&\\
            = e^{-r(T-t)}S_t^{-1} \E^{\Q} \left[ e^{(iu+1)X_T} | F_t \right]
            = e^{-r(T-t)}S_t^{-1} \varphi(u - i)&
        \end{align*}
        $$
            \varphi(-i) = \E^{\Q}\left[ S_T | \F_t \right]= e^{r(T-t)} S_t
        $$
    \end{itemize}
\end{frame}

\begin{frame}{Прайсинг в моделях с известной хар. функцией}
    \begin{block}{Утверждение}
        Пусть $\varphi(u) = \E^\Q \left[e^{iu\ln S_T} | \F_t \right]$ -- условная хар. функция процесса $\ln S_T$. 
        Тогда цены европейских колл-опционов выражаются по формуле:
        $$C_t(T, K) = S_t \Q_S(S_T > K | \F_t) - e^{-r(T-t)} K \Q(S_T > K | \F_t)$$
        где вероятности исполнения вычисляются как:
        \begin{align*}
            &\Q(S_T > K | \F_t) = \dfrac{1}{2} + \dfrac{1}{\pi} \int_{0}^{\infty} \Re \left[ \dfrac{e^{-iu \ln K}\varphi(u)}{iu} du \right]\\
            &\Q_S(S_T > K | \F_t) = \dfrac{1}{2} + \dfrac{1}{\pi} \int_{0}^{\infty} \Re \left[ \dfrac{e^{-iu \ln K}\tilde{\varphi}(u)}{iu} du \right]
        \end{align*}
        и $\tilde{\varphi}(u) = \dfrac{\varphi(u - i)}{\varphi(-i)}$ -- условная хар. функция в мере $\Q_S$.
    \end{block}
\end{frame}

\begin{frame}{Характеристическая функция в модели Хестона}
    \begin{itemize}
    \item Введём $X_t = \ln S_t$
    \item Условная хар. фунцкия 
    $$\varphi(t, x, v; u) = \E^{\Q}\left[ e^{iuX_T} | X_t = x, v_t = v\right]$$
    \item Формула Феймана-Каца:
        \begin{align*}
            &\dfrac{\partial \varphi}{\partial t}
            + \left(r - \frac{1}{2}v\right) 
            \dfrac{\partial \varphi}{\partial x} 
            + \kappa (\theta - v) \dfrac{\partial \varphi}{\partial v} 
            + v \dfrac{1}{2}\dfrac{\partial^2 \varphi}{\partial x^2} 
            + \rho \xi v \dfrac{\partial^2 \varphi}{\partial x \partial v}  
            +  \xi^2 v \dfrac{1}{2}\dfrac{\partial^2 \varphi}{\partial v^2} = 0\\
            &\varphi(T, x, v; u) = e^{iux}
        \end{align*}
    \end{itemize}
\end{frame}

\begin{frame}{Характеристическая функция в модели Хестона}
    Пусть $\tau = T - t$ -- время до погашения.
    \begin{block}{Лемма}Характеристическая функция имеет вид:
        $$\varphi(t, x, v; u) = \exp(C(T-\tau; u) + D(T-\tau; u) v + iux)$$
        где функции $C(\tau; u), D(\tau; u)$ удовлетворяют ОДУ:
        \begin{align*}
            &C' = iur + \kappa \theta D\\
            &D' = -\frac{1}{2}(iu+u^2) + (\rho \xi ui - \kappa)D + \frac{1}{2} \xi^2 D^2\\
            &C(0) = D(0) = 0
        \end{align*}
    \end{block}
\end{frame}

\begin{frame}{Характеристическая функция в модели Хестона}
    \textit{Доказательство}. Подставим выражение для $\varphi$ выше в исходное уравнение, получим:
    \begin{itemize}
        \item $\varphi_{x} = iu\varphi$, $\varphi_{xx} = -u^2 \varphi$
        \item $\varphi_v = D \varphi$, $\varphi_{vv} = D^2 \varphi$
        \item $\varphi_{vx} = iuD\varphi$, $\varphi_t = -\varphi \left( C' + D'v \right)$
    \end{itemize}
    $$
    \varphi \left[ 
            -C' - D'v
            + \left(r - \frac{1}{2}v\right) iu 
            + \kappa (\theta - v) D 
            - v \dfrac{1}{2}u^2 
            + \rho \xi v iuD  
            +  \xi^2 v \dfrac{1}{2}D^2 
    \right] = 0
    $$Зануляя слагаемые при $v^0$ и $v^1$ получим требуемые уравнения.
\end{frame}

\begin{frame}{Характеристическая функция в модели Хестона}
    \begin{itemize}
    \item Для фиксированного \( u \) \( D(\tau, u) \) удовлетворяет уравнению Риккати:
    \[
        D^{\prime} = \alpha + \beta D + \gamma D^2, \quad D(0, u) = 0,
    \]где
    \[
    \alpha = -\frac{iu + u^2}{2}, \quad \beta = \rho\xi iu - \kappa, \quad \gamma = \frac{\xi^2}{2}.
    \]
    \item Корни характеристического уравнения  
    \[
    r_{\pm} = \frac{\beta \pm d}{2}, \quad d = \sqrt{\beta^2 - 4\alpha\gamma} = \sqrt{(\rho\xi iu - \kappa)^2 + \xi^2(iu + u^2)}.
    \]  
    \item Решение  
    \[
    D(\tau) = -\frac{1}{\gamma}\left(\frac{K_1 r_+ e^{r_+\tau} + K_2 r_- e^{r_-\tau}}{K_1 e^{r_+\tau} + K_2 e^{r_-\tau}}\right).
    \]
    \item Из граничных условий:
    \[
        K_1 = 1, \quad K_2 = -\frac{r_+}{r_-}.
        \]  
    \end{itemize}
\end{frame}

\begin{frame}{Характеристическая функция в модели Хестона}
    \begin{itemize}
        \item Решение
    \[
    D(\tau, u) = \frac{\kappa - \rho\xi iu - d}{\xi^2} \left( \frac{1 - e^{-d\tau}}{1 - ge^{-d\tau}} \right), \quad \text{где } g = \frac{r_+}{r_-},
    \]  
    \[
    C(\tau, u) = iur\tau + \frac{\kappa\theta}{\xi^2} \left( (\kappa - \rho\xi iu - d)\tau - 2\ln\left( \frac{1 - ge^{-d\tau}}{1 - g} \right) \right).
    \]
    \end{itemize}
\end{frame}

\begin{frame}{Решение уравнения Риккати}
    Уравнение Риккати: 
    $$
        y' = \alpha + \beta y + \gamma y^2
    $$ 
    Замена $y = -\dfrac{z'}{\gamma z}$приводит к уравнению:
    $$
        z'' - \beta z' + \alpha \gamma z = 0
    $$Характеристическое уравнение:
    $$
        r^2 - \beta r + \alpha \gamma = 0 \to 
        r_{\pm} = \dfrac{\beta \pm d}{2}
    $$где $d = \sqrt{\beta^2 - 4\alpha \gamma}$. Решение:
    $$
        z = K_1 e^{r_+ x} + K_2 e^{r_- x} 
    $$ Обратная замена:
    $$
        y = -\dfrac{1}{\gamma} \left( \dfrac{K_1 r_+ e^{r_+x} + K_2r_-e^{r_-}}{K_1 e^{r_+ x} + K_2 e^{r_- x}} \right)
    $$
\end{frame}

\begin{frame}{Характеристическая функция в модели Хестона}
    Пусть $\tau = T - t$ -- время до погашения. Зафиксируем $u$. Пусть
    \begin{align*}
        d = \sqrt{(\rho\xi iu - \kappa)^2 + \xi^2(iu + u^2)}, 
        \quad g = \frac{\rho\xi iu - \kappa + d}{\rho\xi iu - \kappa - d}
    \end{align*}

    \begin{block}{Лемма}Характеристическая функция имеет вид:
        $$\varphi(t, x, v; u) = \exp(C(T-\tau; u) + D(T-\tau; u) v + iux)$$
        где функции $C(\tau; u), D(\tau; u)$ задаются как:
        \[
        D(\tau, u) = \frac{\kappa - \rho\xi iu - d}{\xi^2} \left( \frac{1 - e^{-d\tau}}{1 - ge^{-d\tau}} \right),
        \]  
        \[
        C(\tau, u) = iur\tau + \frac{\kappa\theta}{\xi^2} \left( (\kappa - \rho\xi iu - d)\tau - 2\ln\left( \frac{1 - ge^{-d\tau}}{1 - g} \right) \right).
        \]
    \end{block}
\end{frame}

\begin{frame}{Результаты}
    \centering
    \makebox[\textwidth]{\includegraphics[width=0.6\textwidth]{7_figs/implied_vol.jpg}}
\end{frame}

\begin{frame}{Связь с моделью локальной волатильности}
    Модель Хестона:
        \begin{align*}
        &dS_t/S_t = r dt + \sqrt{v_t} dW_t\\
        &dv_t = \kappa (\theta - v_t) dt + \xi \sqrt{v_t} dZ_t
    \end{align*}

    \begin{block}{Теорема}
        Рассмотрим одномерный процесс 
        $$
            d\tilde{S}_t/\tilde{S}_t = r dt + \sigma_{loc}(t, \tilde{S}) dB_t
        $$где $B_t$ -- броуновское движение, функция локальной волатильности задаётся как:
        $$
            \sigma_{loc}^2(t, s) = \E^\Q \left[ v_t | S_t=s\right]
        $$Тогда $\tilde{S}_t \overset{d}{=} S_t$
    \end{block}
\end{frame}

\begin{frame}{Связь с моделью локальной волатильности}
    \textit{Доказательство для $r=0$} 
    \begin{itemize}   
    \item Пусть $p(t, S, v)$ -- двумерная плотность процесса $(S_t, v_t)$. 
    \item Уравнение Колмогорова:
        \begin{align*}
            \frac{\partial p}{\partial t} = & -\frac{\partial}{\partial v}[\kappa (\theta - v) p] + \frac{1}{2} \frac{\partial^2}{\partial S^2}[S^2 v p] \\
            & + \frac{1}{2} \frac{\partial^2}{\partial v^2}[\xi^2 v p] + \rho \xi \frac{\partial^2}{\partial S \partial v}[S v p]
        \end{align*}
    \item Одномерная плотность:
    $$
        q(t, S) = \int_{0}^{\infty} p(t, S, v) dv
    $$
    \end{itemize}
\end{frame}

\begin{frame}{Связь с моделью локальной волатильности}
    \begin{itemize}
    \item Интегрируем по $v$ от $0$ до $+\infty$:
        \begin{align*}
        \int_0^\infty \frac{\partial p}{\partial t} dv = & -\int_0^\infty \frac{\partial}{\partial v}[\kappa (\theta - v) p] dv \\
        & + \frac{1}{2} \int_0^\infty \frac{\partial^2}{\partial S^2}[S^2 v p] dv \\
        & + \frac{1}{2} \int_0^\infty \frac{\partial^2}{\partial v^2}[\xi^2 v p] dv \\
        & + \rho \xi \int_0^\infty \frac{\partial^2}{\partial S \partial v}[S v p] dv
        \end{align*}
    \end{itemize}
\end{frame}

\begin{frame}{Связь с моделью локальной волатильности}
    \begin{itemize}
    \item $\int_0^\infty \frac{\partial p}{\partial t} dv = \frac{\partial q}{\partial t}$
    \item $-\int_0^\infty \frac{\partial}{\partial v}[\kappa (\theta - v) p] dv = 0$ (граничные условия)
    \item $\frac{1}{2} \int_0^\infty \frac{\partial^2}{\partial S^2}[S^2 v p] dv = \frac{1}{2} \frac{\partial^2}{\partial S^2} \left[ S^2 \int_0^\infty v p dv \right]$
    \item $\frac{1}{2} \int_0^\infty \frac{\partial^2}{\partial v^2}[\xi^2 v p] dv = 0$ (граничные условия)
    \item $\rho \xi \int_0^\infty \frac{\partial^2}{\partial S \partial v}[S v p] dv = 0$ (граничные условия)
    \end{itemize}
    \end{frame}

\begin{frame}{Связь с моделью локальной волатильности}
    \begin{itemize}
    \item Заметим, что:
        \[
        \int_0^\infty v p(t, S, v) dv = \mathbb{E}[v_t | S_t = S] q(t, S)
        \]
    \item Итоговое уравнение на маргинальную плотность: 
        \[
            \frac{\partial q}{\partial t} = \frac{1}{2} \frac{\partial^2}{\partial S^2} \left( S^2 \mathbb{E}[v_t | S_t = S] q(t, S) \right)
        \]

    \item  Формула для локальной волатильности: 
        \[
        \sigma_{\text{loc}}^2(t, S) = \mathbb{E}[v_t | S_t = S]
        \]

    \end{itemize}
\end{frame}

\end{document}

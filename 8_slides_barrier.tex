\documentclass[aspectratio=169]{beamer}
\usepackage{preamble_beamer}

\newcommand*{\QEDB}{\null\nobreak\hfill\ensuremath{\square}}

\newtheorem*{question*}{Вопрос}
\newtheorem*{claim*}{Утверждение}

\title[Барьерные опционы]{Лекция 8. Барьерные опционы} % The short title appears at the bottom of every slide, the full title is only on the title page

\begin{document}

\begin{frame}
\titlepage
\end{frame}

\begin{frame}{Барьерные опционы}

\begin{itemize}
    \item Модель Блэка-Шоулза:
        \begin{align*}
            dS_t / S_t &= r dt + \sigma dW_t \\
            dB_t &= r B_t dt
        \end{align*}
    \item Накопленный максимум/минимум 
            \begin{align*}
            M_t = \max_{u\leq t} S_u, \quad m_t = \min_{u\leq t} S_u 
           \end{align*}
    \item Выплата $\Phi(S_T)$
\end{itemize}
\begin{block}{Определение}
    Барьерный up-and-out опцион с функцией выплаты $\Phi(S_T)$ и барьером $B$ это дериватив, который в момент времени $T$ 
    платит случайную сумму денег, равную
    $$\Phi(S_T) \cdot \mathbb{I}(M_T < B)
    = \begin{cases}
        \Phi(S_T), M_T < B\\
        0, M_T \geq B
    \end{cases}$$
\end{block}
\end{frame}

\begin{frame}{Барьерные опционы: пример}
    \begin{itemize}
        \item Up-and-out пут-опцион. Функция выплаты $\Phi(S_T) = (S_T - K)^+$
        \item Пэйофф барьерного опциона $(S_T - K)^+ \cdot \mathbb{I}(M_T < B)$
    \end{itemize}
    \centering
    \makebox[\textwidth]{\includegraphics[width=0.8\textwidth]{8_figs/barrer_example.jpg}}
\end{frame}

\begin{frame}{Другие барьерные опционы}
    \begin{itemize}
        \item Up-and-out (UO)
        $$
            \mathrm{Payoff} = \Phi(S_T) \cdot \mathbb{I}(M_T < B)
        $$
        \item Up-and-in (UI)
        $$
            \mathrm{Payoff} = \Phi(S_T) \cdot \mathbb{I}(M_T \geq B)
        $$
        \item Down-and-out (DO)
        $$
            \mathrm{Payoff} = \Phi(S_T) \cdot \mathbb{I}(m_t \geq B)
        $$
        \item Down-and-in (DI)
        $$
            \mathrm{Payoff} = \Phi(S_T) \cdot \mathbb{I}(m_t < B)
        $$
    \end{itemize}
\end{frame}

\begin{frame}{Барьерные опционы: свойства}
    \begin{itemize}
        \item In-out parity
        $$
            V^{UO}(t, S_t; \Phi) + V^{UI}(t, S_t; \Phi) = V(t, S_t; \Phi)
        $$
        \item Оценка сверху:
        \begin{align*}
            &V^{UO}(t, S_t; \Phi) \leq V(t, S_t; \Phi) \\
            &V^{UI}(t, S_t; \Phi) \leq V(t, S_t; \Phi)
        \end{align*}
    \end{itemize}
\end{frame}

\begin{frame}{Уравнение Блэка-Шоулза}
    \begin{itemize}
        \item Пусть $M_t < B$, т.е. барьер не пробит до момента $t$
        \item Пусть $\tau = \inf_{u\geq t} \{ S_u \geq B \}$
        \item $\{M_T \leq B\} \Leftrightarrow \{\tau > T\}$
        \item Стоимость опциона:
        $$
            V^{UO}(t, S_t; \Phi) = \E \left[e^{-r(T-t)} \Phi(S_T) \mathbb{I}(\tau > T) | \F_t \right]
        $$
        \item По формуле Феймана-Каца стоимость удовлетворяет уравнению БШ
        \begin{align*}
            &\dfrac{\partial V^{UO}}{\partial t} + 
            r S \dfrac{\partial V^{UO}}{\partial S}
            + \dfrac{1}{2}\sigma^2 S^2 \dfrac{\partial^2 V^{UO}}{\partial S^2}
            = r V^{UO}, \quad 0 \leq t \leq T, 0 \leq S \leq B \\
            & V^{UO}(t, B) = 0\\
            & V^{UO}(T, S) = \Phi(S) \mathbb{I}(S \leq B)
        \end{align*}
    \end{itemize}
\end{frame}

\begin{frame}{Обрезанные пэйоффы}
    \begin{itemize}
        \item Введём обрезаныне пэйоффы:
        $$
            \Phi^B(S) = \Phi(S) \cdot \mathbb{I}(S \leq B) = \begin{cases}
                \Phi(S), S \leq B \\
                0, S > B
            \end{cases}
        $$
        \item Стоимость европейских опционов:
        $$
            V(t, S_t; \Phi^B) =  \E \left[e^{-r(T-t)} \Phi^B(S_T) | \F_t \right]
        $$
        \item Классическое уравнение БШ:
        \begin{align*}
            &\dfrac{\partial V}{\partial t} + 
            r S \dfrac{\partial V}{\partial S}
            + \dfrac{1}{2}\sigma^2 S^2 \dfrac{\partial^2V}{\partial S^2}
            = r V, \quad 0 \leq t \leq T, 0 \leq S < \infty \\
            & V(T, S) = \Phi(S) \mathbb{I}(S \leq B)
        \end{align*}
    \end{itemize}
\end{frame}

\begin{frame}{Замена переменной}
    \begin{itemize}
        \item Замена переменных $X_t = \log S_t$
        \item Логарифмический барьер $b = \log B$
        \item Уравнение БШ в новых координатах $v(t, x) = V(t, e^x; \Phi^B)$:
        \begin{align*}
            &\dfrac{\partial v}{\partial t} + 
            \gamma \dfrac{\partial v}{\partial x}
            + \dfrac{1}{2}\sigma^2 \dfrac{\partial^2 v}{\partial x^2}
            = r v \\          
            & v(T, x) = \Phi^B(e^x)
        \end{align*}где $\gamma = r - \dfrac{1}{2}\sigma^2$
    \end{itemize}
\end{frame}

\begin{frame}{Метод отражения}
    \begin{itemize}
        \only<1>{
        \item Пусть $\gamma = 0$, т.е. $r=0.5 \sigma^2$
        \item Уравнение БШ в новых координатах $v(t, x) = V(t, e^x; \Phi^B)$:
        \begin{align*}
            &\dfrac{\partial v}{\partial t}
            + \dfrac{1}{2}\sigma^2 \dfrac{\partial^2 v}{\partial x^2}
            = r v
        \end{align*}
        \item В силу симметрии уравнения, $v(t, 2b - x)$ тоже решение.
        \item Терминальные условия при $x < b$:
        $$
            v(T, 2b - x) = \Phi^B(e^{2b - x}) = \Phi^B\left(\frac{B^2}{S}\right) = 0  
        $$}
    \only<2>{
        \item Отсюда $g(t, x) = v(t, x) - v(t, 2b - x)$ удовлетворяет уравнению:
        \begin{align*}
            &\dfrac{\partial g}{\partial t}
            + \dfrac{1}{2}\sigma^2 \dfrac{\partial^2 g}{\partial x^2}
            = r v, x \leq b\\
            & g(T, b) = 0\\           
            & g(T, x) = \Phi^B(e^x)
        \end{align*}
        т.е. $V^{UO}(t, S; \Phi) = g(t, \log S)$ -- стоимость барьерного опциона.}
    \end{itemize}
\end{frame}

\begin{frame}{Метод отражения}
    \begin{block}{Теорема}
        Пусть $r = 0.5 \sigma^2$, пусть $V^{UO}(t, S_t; \Phi)$ -- стоимость барьерного up-and-out опциона. Тогда:
        \begin{align*}
            V^{UO}(t, S_t; \Phi) = \begin{cases}
                0, M_t \geq B \\
                V(t, S_t; \Phi^B) - V\left( t, \frac{B^2}{S_t}; \Phi^B \right), M_t < B
            \end{cases}
        \end{align*}
        где $V(t, S_t; \Phi^B)$ -- стоимость европейского опциона с обрезанным пэйоффом.
    \end{block}
\end{frame}

\begin{frame}{Пример}
    \begin{itemize}
        \item Рассмотрим up-and-out пут-опцион со страйком $K$ и барьером $B$
        \item Соответсвующий обрезанный пэйофф: 
        \begin{align*}
            \Phi^B(S_T) &= (K-S_T)^+ \mathbb{I}(S_T \leq B)
            = (K-S_T) \mathbb{I}(S_T \leq K) \mathbb{I}(S_T \leq B) = \\
            &= 
             K \mathbb{I} (S_T \leq \min(K, B)) - S_T \mathbb{I} (S_T \leq \min(K, B))
        \end{align*}
    \end{itemize}
        \centering
        \makebox[\textwidth]{\includegraphics[width=0.8\textwidth]{8_figs/barrier_put_payoff.jpg}}
\end{frame}

\begin{frame}{Пример: продолжение}
    \begin{itemize}
        \only<1>{\item Цена европейского пэйоффа:
        $$\mathrm{Put}(t, S_t; K, B) = 
            Ke^{-r(T-t)} \mathcal{N}(-d_1 + \sigma \sqrt{T-t}) - S_t \mathcal{N}(-d_1)
        $$ где 
        \begin{align*}
            &d_1 = \frac{\ln(S_t / \min(B, K)) + (r + 0.5 \sigma^2)(T-t)}{\sigma \sqrt{T-t}}
        \end{align*}}
        \only<1->{
        \item Цена барьерного опциона:
        $$
            V^{UO}(t, S_t; \Phi) = \mathrm{Put}(t, S_t; K, B) - 
            \mathrm{Put}\left(t, \dfrac{B^2}{S_t}; K, B\right)
        $$}
    \end{itemize}
    \only<2->{
            \centering
        \makebox[\textwidth]{\includegraphics[width=0.8\textwidth]{8_figs/barrier_put_prices_vs_spot.jpg}}
    }
\end{frame}

\begin{frame}{Метод отражения: общий случай}
    \begin{itemize}
        \only<1>{\item Пусть $\gamma \neq 0$. Уравнение БШ в новых координатах $v(t, x) = V(t, e^x; \Phi^B)$:
        \begin{align*}
            &\dfrac{\partial v}{\partial t}
            + \gamma \dfrac{\partial v}{\partial x}
            + \dfrac{1}{2}\sigma^2 \dfrac{\partial^2 v}{\partial x^2}
            = r v
        \end{align*}\item Пусть $v(t, x)$ -- решение. Ищем второе решение $h(t, x) = e^{\alpha (x-b)} v(t, 2b - x)$
        \item Производные:
        \begin{align*}
            & h_t = e^{\alpha (x-b)} v_t, \quad h_x = e^{\alpha (x-b)}(\alpha v - v_x)\\
            & h_{xx} = e^{\alpha (x-b)} (v_{xx} - 2\alpha v_x + \alpha^2 v)
        \end{align*}
        }
        \only<1->{\item Подставляем в уравнение:
        \begin{align*}
            0 \overset{?}{=} &h_t + \gamma h_x + \frac{1}{2} \sigma^2 h_{xx} - rh =
            \\ = 
            &e^{\alpha (x-b)} \left( v_t + v_x (-\gamma - \alpha \sigma^2) + 
            \frac{1}{2} \sigma^2 v_{xx} - v(r - \gamma \alpha - \frac{\alpha^2\sigma^2}{2}) \right)
        \end{align*}}
        \only<2->{\item При $\alpha = -\frac{2\gamma}{\sigma^2}$ правая часть равна нулю, поэтому $g(t, x)$ является решением.

        \item Отсюда:
        $$
            g(t, x) = v(t, x) - e^{-\frac{2\gamma}{\sigma^2} (x-b)} v(t, 2b - x)
        $$ является решением уравнения:
        \begin{align*}
            &\dfrac{\partial g}{\partial t}
            + \dfrac{1}{2}\sigma^2 \dfrac{\partial^2 g}{\partial x^2}
            = r v, x \leq b\\
            & g(T, b) = 0\\           
            & g(T, x) = \Phi^B(e^x)
        \end{align*}
        }
    \end{itemize}
\end{frame}

\begin{frame}{Метод отражения: общий случай}
    \begin{block}{Теорема}
        Пусть $\gamma = r - 0.5 \sigma^2$, $V^{UO}(t, S_t; \Phi)$ -- стоимость барьерного up-and-out опциона. Тогда:
        \begin{align*}
            V^{UO}(t, S_t; \Phi) = \begin{cases}
                0, M_t \geq B \\
                V(t, S_t; \Phi^B) - \left(\dfrac{B}{S_t}\right)^{\frac{2\gamma}{\sigma^2}}V\left( t, \frac{B^2}{S_t}; \Phi^B \right), M_t < B
            \end{cases}
        \end{align*}
        где $V(t, S_t; \Phi^B)$ -- стоимость европейского опциона с обрезанным пэйоффом.
    \end{block}
\end{frame}

\begin{frame}{Контракты типа up-and-in}
    \begin{itemize}
        \item Обрезанные снизу пэйоффы:
        $$
            \Phi_B(S) = \Phi(S) \mathbb{I}(S > B)
        $$
        $$\Phi = \Phi_B + \Phi^B$$
        \item Европейский контракт:
        $$
            V(t, S_t; \Phi) = V(t, S_t; \Phi_B) + V(t, S_t; \Phi^B)
        $$
        \item In-out parity:
        $$
            V^{UO}(t, S_t; \Phi) + V^{UI}(t, S_t; \Phi) = V(t, S_t; \Phi)
        $$
        \begin{align*}
             &V^{UI}(t, S_t; \Phi) = 
             V(t, S_t; \Phi) - V^{UO}(t, S_t; \Phi) = \\
             = &V(t, S_t; \Phi_B) + V(t, S_t; \Phi^B)
             - V(t, S_t; \Phi^B) + \left(\frac{B}{S_t}\right)^{\frac{2\gamma}{\sigma^2}}V\left( t, \frac{B^2}{S_t}; \Phi^B \right)
             = \\ 
             = &V(t, S_t; \Phi_B) + \left(\frac{B}{S_t}\right)^{\frac{2\gamma}{\sigma^2}}V\left( t, \frac{B^2}{S_t}; \Phi^B \right)
        \end{align*}
    \end{itemize}
\end{frame}

\begin{frame}{Контракты типа up-and-in}
    \begin{block}{Теорема}
        Пусть $\gamma = r - 0.5 \sigma^2$, $V^{UI}(t, S_t; \Phi)$ -- стоимость барьерного up-and-in опциона. Тогда:
        \begin{align*}
            V^{UI}(t, S_t; \Phi) = \begin{cases}
                0, M_t \geq B \\
                V(t, S_t; \Phi_B) + \left(\dfrac{B}{S_t}\right)^{\frac{2\gamma}{\sigma^2}}V\left( t, \frac{B^2}{S_t}; \Phi^B \right), M_t < B
            \end{cases}
        \end{align*}
        где $V(t, S_t; \Phi_B)$ -- стоимость европейского опциона с обрезанным пэйоффом.
    \end{block}
\end{frame}

\end{document}

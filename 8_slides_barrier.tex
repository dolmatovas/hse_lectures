\documentclass[aspectratio=169]{beamer}
\usepackage{preamble_beamer}

\newcommand*{\QEDB}{\null\nobreak\hfill\ensuremath{\square}}

\newtheorem*{question*}{Вопрос}
\newtheorem*{claim*}{Утверждение}

\title[Модели стохастической волатильности]{Лекция 7. Модель Хестона} % The short title appears at the bottom of every slide, the full title is only on the title page

\begin{document}

\begin{frame}
\titlepage
\end{frame}

\begin{frame}{Постановка задачи}

\begin{itemize}
    \item Модель Блэка-Шоулза:
        \begin{align*}
        \frac{dS_t}{S_t} &= r dt + \sigma dW_t
        \end{align*}
    \item Верхний барьер $B$
    \item Накопленный максимум $M_t = \max_{u\leq t} S_u$
    \item Выплата: 
        $$\Phi(S_T, M_T) = \Phi(S_T) \cdot \mathbb{I}(M_T < B)
            = \begin{cases}
                \Phi(S_T), M_T < B\\
                0, M_T \geq B
            \end{cases}$$
\end{itemize}
\end{frame}

\begin{frame}{Уравнение Блэка-Шоулза}
    \begin{itemize}
        \item Пусть $M_t < B$, т.е. барьер не пробит до момента $t$
        \item Пусть $\tau = \inf_{u\geq t} \{ S_u \geq B \}$
        \item Пэйофф: $\Phi(S_T) \mathbb{I}(\tau > T)$
        \item Стоимость опциона:
        $$
            V^{UO}(t, S_t) = \E e^{-r(T-t)} \Phi(S_T) \mathbb{I}(\tau > T)
        $$
        \item По формуле Феймана-Каца стоимость удовлетворяет уравнению БШ
        \begin{align*}
            &\dfrac{\partial V^{UO}}{\partial t} + 
            r S \dfrac{\partial V^{UO}}{\partial S}
            + \dfrac{1}{2}\sigma^2 S^2 \dfrac{\partial^2 V^{UO}}{\partial S^2}
            = r V^{UO}, \quad 0 \leq t \leq T, 0 \leq S \leq B \\
            & V^{UO}(t, B) = 0\\
            & V^{UO}(T, S) = \Phi(S) \mathbb{I}(S \leq B)
        \end{align*}
    \end{itemize}
\end{frame}

\begin{frame}{Обрезанные пэйоффы}
    \begin{itemize}
        \item Введём обрезаныне пэйоффы:
        $$
            \Phi^B(S) = \Phi(S) \cdot \mathbb{I}(S \leq B) = \begin{cases}
                \Phi(S), S \leq B \\
                0, S > B
            \end{cases}
        $$
    \end{itemize}
\end{frame}

\begin{frame}{Замена переменной}
    \begin{itemize}
        \item Замена переменных $X_t = \log S_t$
        \item Логарифмический барьер $b = \log B$
        \item Уравнение БШ в новых координатах $v^{UO}(t, x) = V^{UO}(t, e^x)$:
        \begin{align*}
            &\dfrac{\partial v^{UO}}{\partial t} + 
            \gamma \dfrac{\partial v^{UO}}{\partial x}
            + \dfrac{1}{2}\sigma^2 \dfrac{\partial^2 v^{UO}}{\partial x^2}
            = r v^{UO} \\           
            & v^{UO}(t, b) = 0\\
            & v^{UO}(T, x) = \Phi^B(e^x)
        \end{align*}где $\gamma = r - \dfrac{1}{2}\sigma^2$
    \end{itemize}
\end{frame}

\begin{frame}{Метод отражения}
    \begin{itemize}
        \item Пусть $\gamma = 0$, т.е. $r=0.5 \sigma^2$
        \item Пусть $v(t, x)$ -- стоимость европейского опциона с пэйоффом $\Phi^B(e^x)$
        \item Функция $v(t, x)$ удовлетворяет уравнению:
        \begin{align*}
            &\dfrac{\partial v}{\partial t}
            + \dfrac{1}{2}\sigma^2 \dfrac{\partial^2 v}{\partial x^2}
            = r v
        \end{align*}
        \item В силу симметрии $v(t, 2b - x)$ тоже решение. 
        \item Терминальное условие:
        $
             v(T, 2a - x) = \Phi^B(e^{2b - x}) = 0
        $ при $x \leq b$. 
        \item Отсюда $g(t, x) = v(t, x) - v(t, 2b - x)$ удовлетворяет уравнению:
        \begin{align*}
            &\dfrac{\partial g}{\partial t}
            + \dfrac{1}{2}\sigma^2 \dfrac{\partial^2 g}{\partial x^2}
            = r v \\
            & g(T, b) = 0\\           
            & g(T, x) = \Phi^B(e^x)
        \end{align*}
        т.е. $v^{UO}(t, x) = g(t, x)$ -- стоимость барьерного опциона.
    \end{itemize}
\end{frame}

\begin{frame}{Метод отражения}
    \begin{block}
        $$
            V^{UO}(t, S_t; \Phi) = V(t, S_t; \Phi^L) - V\left(t, \dfrac{L^2}{S_t}; \Phi^L\right)
        $$
    \end{block}
\end{frame}

\begin{frame}{Пример}
    \begin{itemize}
        \item Пусть $\Phi(S_T) = (K-S_T)^+$, $K \leq L$
        \item $\Phi^L(S_T) = \Phi(S_T)$
        \item 
        $$
            P^{UO}(t, S_t) = P(t, S_t) - P(t, \dfrac{L^2}{S_t})
            = \ldots
        $$
    \end{itemize}
\end{frame}

\begin{frame}{Переход к логарифмическим координатам}
Введём: $X_t = \log S_t$

По формуле Ито:
\begin{align*}
dX_t &= \left(r - \frac{1}{2}\sigma^2\right) dt + \sigma dW_t
\end{align*}

При $r = \frac{1}{2}\sigma^2$ получаем:
\begin{align*}
dX_t &= \sigma dW_t
\end{align*}

Барьер в новых координатах: $a = \log B$
\end{frame}

\begin{frame}{123}
    \begin{itemize}
        \item Введём обрезанный пэйофф:
        $$
            \Phi^L(S) = \Phi(S) \cdot \mathbb{I} (S < L)
        $$
        \item В новых координатах:
        $$
            \Phi^L(e^x) = \Phi(e^x) \cdot \mathbb{I} (x < a)
        $$
    \end{itemize}
\end{frame}

\begin{frame}{Уравнение Блэка-Шоулса}
Цена европейского опциона $C(t,x)$ с пэйоффом $\Phi^L(e^x)$ удовлетворяет:
\begin{align*}
\frac{\partial C}{\partial t} + \frac{1}{2}\sigma^2 \frac{\partial^2 C}{\partial x^2} + \left(r - \frac{1}{2}\sigma^2\right) \frac{\partial C}{\partial x} - rC = 0
\end{align*}

При $r = \frac{1}{2}\sigma^2$:
\begin{align*}
\frac{\partial C}{\partial t} + \frac{1}{2}\sigma^2 \frac{\partial^2 C}{\partial x^2} - rC = 0
\end{align*}
\end{frame}

\begin{frame}{Симметрия уравнения}
Уравнение:
\begin{align*}
\\frac{\partial C}{\partial t} + \frac{1}{2}\sigma^2 \frac{\partial^2 C}{\partial x^2} - rC = 0
\end{align*}

\textbf{Симметрия}: уравнение инвариантно относительно замены $x \to -x$

Если $C(t,x)$ - решение, то $C(t,2a - x)$ тоже решение.
\end{frame}

\begin{frame}{Метод отражения}
Построим новую функцию:
\begin{align*}
G(t,x) = C(t,x) - C(t,2a-x)
\end{align*}

\textbf{Свойства}:
\begin{itemize}
\item $G(t,x)$ - решение уравнения (линейная комбинация решений)
\item $G(t,a) = C(t,a) - C(t,a) = 0$ (условие на барьере)
\end{itemize}
\end{frame}

\begin{frame}{Начальные условия}
При $t = T$ (экспирация):
\begin{align*}
G(T,x) &= C(T,x) - C(T,2a-x) = \Phi^L(e^x) - \Phi^L(e^{2a - x})
= \Phi^L(e^x)
\end{align*}так как $2a - x > a$ при $x < a$.

\textbf{Проверим выполнение условий барьерного опциона:}
\end{frame}

\begin{frame}{Итог}
    Пусть $C^L(t, x)$ -- цена опциона с пэйоффом $\Phi^L(x)$. Тогда цена барьерного опциона задаётся как:
    $$
        C^{LO}(t, x) = \begin{cases}
            C^L(t, x) - C^L(t, 2a - x), \text{при } a < M_t\\
            0, \text{при} a \geq M_t
        \end{cases}
    $$
\end{frame}


\end{document}

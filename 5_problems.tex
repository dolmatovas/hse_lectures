\documentclass[12pt]{article}

\usepackage{preamble_problemset}
\usepackage{amsmath}

\begin{document}
\noindent Количественные финансы, осень 2025\hfill Семинар 4\\
\today

\hrulefill

\begin{problem}
    Пусть $(\Omega, \mathcal{F}, \mathbb{P})$ -- вероятностное пространство, $W_t$ -- броуновское движение. Показать, что для любой эквивалентной меры 
    $\mathbb{Q} \sim \mathbb{P}$ выполнено:
    $$
        [W]_t = t
    $$ $\mathbb{Q}$-почти наверное.
\end{problem}

\begin{problem}
    Найти стоимость дериватива $\Phi(S_T) = S_T \mathbb{I}(S_T \geq K)$ с помощью замены меры при $N_t = S_t$.
\end{problem}

\begin{problem}
    Рассмотрим рынок с тремя активами:
    \begin{align*}
        &dB_t = 0\\
        &dS_t^1 = S_t^1 \sigma_1 dW_t^1 \\
        &dS_t^2 = S_t^2 \sigma_2 dW_t^2
    \end{align*}где $W_t^1, W_t^2$ -- два броуновских движения с корреляцией $\rho$. Найти стоимость обменного опциона с пэйоффом:
    $$
        \Phi(S_T^1, S_T^2) = (S_T^1 - S_T^2)^+
    $$
\end{problem}

\begin{problem}
    Реплицировать с помощью ванильных опционов пэйофф:
    $$
        \Phi(S_T) = g(S_T)
    $$где $g(x)$ -- гладкая финитная функция. Найти стоимость дериватива с таким пэйоффом.
\end{problem}

\begin{problem}[Variance swap]
    Пусть $dX_t = X_t \sigma_t dB_t$ -- процесс Ито, $\sigma_t$ -- согласованный процесс.
    \\ Покажите, что:
    $$
        \int_0^T \sigma^2_t dt = -2\ln \dfrac{X_T}{X_0} + \int_0^T \dfrac{2}{X_t}dX_t
    $$
\end{problem}

\end{document}

\documentclass[12pt]{article}

\usepackage{preamble_problemset}


\begin{document}
\noindent Количественные финансы, осень 2025\hfill ДЗ 2\\
\today

\hrulefill

\begin{problem}[Стохастический интеграл 2]
    Пусть $B_t$ -- броуновское движение.  Введём forward-looking стохастический интеграл по формуле:
    $$
        \int_0^t g_s \circ dB_s := \lim_{\delta \to 0} \sum_{k=0}^{n-1} g_{t_{k+1}} (B_{t_{k+1}} - B_{t_k})
    $$где $0=t_0 < \ldots < t_n = t$ -- разбиение $[0, t]$, предел берётся по всем разбиениям при диаметре $\delta \to 0$. Вычислить
    $$
        2 \int_0^t B_s \circ dB_s
    $$Сравнить ответ с интегралом Ито.
\end{problem}

\begin{problem}
    Доказать формулу Ито для процесса Ито.
\end{problem}

\begin{problem}
    При каком $\alpha$ процесс $X_t = e^{\alpha t + \sigma B_t}$ является мартингалом? Решить с помощью леммы Ито.
\end{problem}
 
\begin{problem}
    Пусть $X_t = B_t^4 + f(t) B_t^2 + g(t)$, где $B_t$ -- броуновское движение, $f(t), g(t)$ -- детерминированные функции. 
    \\ При каких $f, g$ процесс $X_t$ является мартингалом? 
\end{problem}

\end{document}

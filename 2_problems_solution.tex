\documentclass[12pt]{article}

\usepackage{preamble_problemset}


\begin{document}
\noindent Количественные финансы, осень 2025\hfill Семинар 2\\
\today

\hrulefill

\begin{problem}
    Докажите, что у броуновского движения почти наверное бесконечная полная вариация.
\end{problem}

\textit{Доказательство}
\begin{itemize}
    \item От противного. Пусть первая вариация равна $V < \infty$.
    \item Пусть $0 = t_0 < t_1 < \ldots < t_n = T$ -- произвольное разбиение с диаметром $\delta$: 
        $$\delta = \max_k \{t_{k+1} - t_k\}$$ 
    \item Вычислим сумму из определения квадратичной вариации: $$
        \sum_{k=0}^{n-1} \left[B_{t_{k+1}} - B_{t_k}\right]^2 \leq 
        \max_{k} |B_{t_{k+1}} - B_{t_k}| \cdot \sum_{k=0}^{n-1} |B_{t_{k+1}} - B_{t_k}|
        \leq \max_{k} |B_{t_{k+1}} - B_{t_k}| \cdot V
    $$
    \item В силу непрерывности $\max_{k} |B_{t_{k+1}} - B_{t_k}|\to 0$ при $\delta \to 0$
    \item Отсюда квадратичная вариация стремится к нулю, противоречие. \qed
\end{itemize}

\begin{problem}
    Пусть 
    $$
        X_t = \mu t + \sigma B_t
    $$
    Пусть $a, b > 0$. Пусть $\tau = \inf_{t \geq 0} \{t: X_t = a \lor X_t = -b\}$.
    \\ Найти $\mathbb{P}(X_{\tau} = a), \E \tau$.
\end{problem}

\textit{Решение}
\begin{itemize}
    \item Пусть $\mathbb{P}(X_{\tau} = a) = p, \mathbb{P}(X_{\tau} = b) = 1-p$
    \item Пусть функция $g(x)$ такая, что процесс $Y_t = g(X_t)$ -- мартингал.
    \item Тогда с одной стороны:
    $$
        \E Y_{\tau} = \E Y_0 = g(0)
    $$С другой стороны:
    $$
    \E Y_{\tau} = p \cdot g(a) + (1-p) \cdot g(-b) = p(g(a) - g(-b)) + g(-b)
    $$Отсюда:
    $$
        p = \dfrac{g(0) - g(-b)}{g(a) - g(-b)}
    $$
    \item Формула Ито для процесса $Y_t$:
    $$
        dY_t = g'(X_t) dX_t + 0.5 g''(X_t) dX_t^2 = \left( g'(X_t) \mu + 0.5 \sigma^2 g''(X_t) \right) dt + g'(X_t) \sigma dB_t
    $$
    \item Хотим, чтобы $Y_t$ был мартингалом, отсюда уравнение на $g$:
    $$
        g' \mu = -0.5 \sigma^2 g''
    $$
    \item Замена переменых: $g' = u$
    $$
        u' = -\dfrac{2 \mu}{\sigma^2} u \to u = C \exp\left(-\dfrac{2 \mu}{\sigma^2}x\right)
    $$
    $$
        g = C_1 + C_2 \exp\left(-\dfrac{2 \mu}{\sigma^2}x\right)
    $$Положим для простоты $C_1 = 0, C_2 = 1$. Итого решение:
    $$
        p = \dfrac{g(0) - g(-b)}{g(a) - g(-b)} = \dfrac{1 - \exp(\dfrac{2 \mu}{\sigma^2}b)}{\exp(-\dfrac{2 \mu}{\sigma^2}a)-\exp(\dfrac{2 \mu}{\sigma^2}b)}
    $$
    \item При $\mu = 0$ по правилу Лопиталя получим:
    $$
        p(\mu = 0.5) = \dfrac{b}{a+b}
    $$
    \item $\E \tau$:
    $$
        \E X_t = \mu \E \tau + \sigma \E B_{\tau} = \mu \E \tau \to \E \tau = \dfrac{\E X_t}{\mu}
    $$
    \item В свою очередь:
    $$
       \E X_t = p a - (1-p)b = p(a + b) - b = \ldots
    $$
\end{itemize}

 
\end{document}

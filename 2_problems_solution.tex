\documentclass[12pt]{article}

\usepackage{preamble_problemset}


\begin{document}
\noindent Количественные финансы, осень 2025\hfill ДЗ к лекции 2\\
\today

\hrulefill

\begin{problem}
    Докажите, что у броуновского движения почти наверное бесконечная полная вариация.
\end{problem}

\textit{Доказательство}
\begin{itemize}
    \item От противного. Пусть первая вариация равна $V < \infty$.
    \item Пусть $0 = t_0 < t_1 < \ldots < t_n = T$ -- произвольное разбиение с диаметром $\delta$: 
        $$\delta = \max_k \{t_{k+1} - t_k\}$$ 
    \item Вычислим сумму из определения квадратичной вариации: $$
        \sum_{k=0}^{n-1} \left[B_{t_{k+1}} - B_{t_k}\right]^2 \leq 
        \max_{k} |B_{t_{k+1}} - B_{t_k}| \cdot \sum_{k=0}^{n-1} |B_{t_{k+1}} - B_{t_k}|
        \leq \max_{k} |B_{t_{k+1}} - B_{t_k}| \cdot V
    $$
    \item В силу непрерывности $\max_{k} |B_{t_{k+1}} - B_{t_k}|\to 0$ при $\delta \to 0$
    \item Отсюда квадратичная вариация стремится к нулю, противоречие. \qed
\end{itemize}
 
\end{document}

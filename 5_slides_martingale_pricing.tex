\documentclass[aspectratio=169]{beamer}
\usepackage{preamble_beamer}

\newcommand*{\QEDB}{\null\nobreak\hfill\ensuremath{\square}}

\title[Мартингальная теория прайсинга]{Лекция 5. Мартингальная теория прайсинга} % The short title appears at the bottom of every slide, the full title is only on the title page


\begin{document}

\begin{frame}
\titlepage 
\end{frame}

\begin{frame}{Рекап прошлой лекции} 
\begin{itemize} 
    \item Динамика самофинансируемого портфеля.
    \item Случайные платёжные обязательства
    \item Безарбитражность и реплицируемость.
    \item Уравнение Блэка-Шоулза.
    \item Европейский колл-опцион, стоимость и греки
\end{itemize}
\end{frame}

\begin{frame}{План лекции}
    \begin{itemize}
        \item Эквивалентные меры.
        \item Производная Радона-Никодима. Формула замены Меры.
        \item Теорема Гирсанова: замена дрифта у броуновского движения.
        \item Фундаментальные теоремы финансов.
        \item Полнота и безарбитражность многомерной модели Блэка-Шоулза. 
    \end{itemize}
\end{frame}

% \begin{frame}{План лекции}
%     \begin{itemize}
%         \item Про меры. Теорема Гирсанова
%         \item Модель БШ. Риск-нейтральная мера. Определение.
%         \item Модель БШ. Риск-нейтральная мера. Существование.
%         \item Безарбитражность как следствие существование риск-нейтральной меры.
%         \item Полнота. Определение. Уже доказали для простых платёжных обязательств.
%         \item Набросок доказательства для произвольных. Пусть модель безарбитражна. 
%         Пусть r=0. Произвольный функционал от БД можно представить как интеграл Ито. 
%         Разложение фурье, полнота базисных функций. Частный случай для процессов Ито, заданных СДУ.
%         \item Многомерная модель. БШ или произвольная? Давайте БШ. Куда деть банковский счёт? Для простоты нулевая ставка?
%         \item Банковский счёт как numeraire. Риск-нейтральная мера. Определение. 
%         \item Условие на дрифты БД. 
%         \item Уравнение на риск-премию. Теорема существования и единственности риск-нейтральной меры. Мета-теорема.
%         \item Первая фундаментальная теорема для многомерного БШ. Условия в терминах риск-премии.
%         \item Тривиальный пример неполного рынка. K > N.
%         \item Вторая фундаментальная теорема для многомерного БШ. Условия в терминах риск-премии. 
%     \end{itemize}
% \end{frame}


\begin{frame}{Абсолютная непрерывность мер}
    Пусть $(\Omega, \F)$ -- измеримое пространство (множество с $\sigma$-алгеброй). Пусть $\Q, \PP$ -- меры на $(\Omega, \F)$.

    \begin{block}{Определение}
        Мера $\Q$ абсолютно непрерывна относительно $\PP$, если $\forall A \in \F$:
        $$
            \PP(A) = 0 \to \Q(A) = 0
        $$Обозначение $\Q \ll \PP$
    \end{block}

    \begin{block}{Определение}
        Мера $\Q$ эквивалентна $\PP$, если $\Q \ll \PP, \PP \ll \Q$. Обозначение $\Q \sim \PP$
    \end{block}
\end{frame}

\begin{frame}{Абсолютная непрерывность мер: примеры}
    Примеры:    
    \begin{itemize}
            \item Пусть $\Omega = \{1, \ldots, N\}, \F=2^{\Omega}$. Тогда:
                $$
                    \Q \ll \PP \Leftrightarrow \PP(\{n\}) = 0 \to \Q(\{n\}) = 0
                $$
        \end{itemize}
        \pause
        \begin{itemize}
        \item Пусть $\Omega=\R, \F = \mathcal{B}(\R)$. Пусть меры $\PP, \Q$ заданы плотностями $p(x), q(x)$, т.е.
        $$
            \PP(A) = \int_A p(x) dx, \; \Q(A) = \int_A q(x) dx
        $$
        Тогда
        $$
            \Q \ll \PP \Leftrightarrow \mathrm{supp}(p) \subseteq\mathrm{supp}(q)
        $$
        \end{itemize}
\end{frame}

\begin{frame}{Производная Радона-Никодима}
    Пусть $\PP$ -- мера, $f \in \F$ -- измеримая фунцкия, $f(\omega)\geq 0$. Определим меру $\Q$ по формуле:
    $$
        \Q(A) = \int_A f(\omega) d\PP(\omega), \; A \in \F
    $$
    
    Тогда $\Q$ -- мера и $\Q \ll \PP$. 
    \pause 
    Верно и обратное:
    \begin{block}{Теорема Радона-Никодима}
        Пусть $\Q \ll \PP$. Тогда $\exists f \in \F, f(\omega) \geq 0$:
        $$
            \Q(A) = \int_A f(\omega) d\PP(\omega)
        $$Обозначение: $f(\omega) = \dfrac{d\Q}{d\PP}$
    \end{block}
\end{frame}

\begin{frame}{Производная Радона-Никодима: примеры}
    \begin{itemize}
        \item Пусть $\Omega = \{1, \ldots, N\}, \F=2^{\Omega}$, $\Q \ll \PP$.
        Тогда:
            $$
                \dfrac{d\Q}{d\PP}(n) = \dfrac{\Q(\{n\})}{\PP(\{n\})} \cdot \mathbb{I}(\PP(\{n\}) \neq 0)
            $$при $n \in \Omega$.
        \pause
        \item Пусть $\Omega=\R, \F = \mathcal{B}(\R)$. Пусть меры $\PP \gg \Q$ заданы плотностями $p(x), q(x)$. 
        Тогда:
            $$
                \dfrac{d\Q}{d\PP}(x) = \dfrac{q(x)}{p(x)} \cdot \mathbb{I}(p(x) \neq 0)
            $$
    \end{itemize}
\end{frame}

\begin{frame}{Производная Радона-Никодима: свойства}
    Пусть 
    \begin{itemize} 
        \item $(\Omega, \F)$ -- измеримое пространство
        \item $\PP \sim \Q$ -- вероятностные меры
        \item $f=\dfrac{d\Q}{d\PP}$ -- производная Радона-Никодима 
        \item $\xi \in \F$ -- случайная величина.
    \end{itemize}
    \pause Тогда:
    \begin{itemize}
        \item $\Q(A) = \int_{A} f(\omega) d\PP(\omega)$
        \item $\E^{\Q} \xi = \int_{\Omega} \xi(\omega) dQ(\omega) = \int_{\Omega} \xi(\omega) f(\omega) d\PP(\omega)
        = \E^{\PP} \left[\xi \cdot f\right]$ 
        \item $\E^{\Q} 1 = \E^{\PP} f = 1$
    \end{itemize}
\end{frame}

\begin{frame}{Замена меры: пример}
    Пусть $\Omega=\R, \F = \mathcal{B}(\R)$, мера $\PP$ задана плотностью $p(x) = \dfrac{\exp(-0.5x^2)}{\sqrt{2\pi}}$.
    \\ Найти меру, относительно которой с.в. $\xi(x) = x$ имеет распределение $\mathcal{N}(a, 1)$.
    \pause
    \begin{itemize}
        \item Плотность новой меры $\Q$:
        $$q(x) = \dfrac{\exp(-0.5(x-a)^2)}{\sqrt{2\pi}}$$
        \item Производная Радона-Никодима:
        $$
            f(x) = \dfrac{q}{p} = \exp(ax - 0.5 a^2)
        $$
        \item Проверка:
        \begin{align*}
            &\E^{\Q} \xi = \E^{\PP} \xi \cdot f(\xi)
            = \dfrac{1}{\sqrt{2\pi}} \int_{\R} x \cdot\exp(ax - 0.5 a^2) \exp(-0.5x^2) dx = \\
            &= \dfrac{1}{\sqrt{2\pi}} \int_{\R} x \cdot\exp(-0.5(x-a)^2) dx = a
        \end{align*}
    \end{itemize}
\end{frame}

\begin{frame}{Замена меры для процессов}
    \begin{block}{Теорема}
        Пусть $(\Omega, \F, \PP)$ -- вероятностное пространство, $\Q \sim \PP$, $\Lambda = \dfrac{d\Q}{d\PP}$, $\Lambda_t = \E^{\PP}\left[ \Lambda | \F_t \right]$. Тогда:
        $$
            \Lambda_t = \dfrac{d\Q_t}{d\PP_t} \in \F_t
        $$где $\Q_t, \PP_t$ -- ограничение мер на $\sigma$-алгебру $\F_t$. 
    \end{block}
    \textit{Доказательство}. Пусть $A_t \in \F_t$. По определению производной Радона-Никодима:
    $$
        \Q(A_t) = \E^{\PP} \left[\Lambda \cdot \mathbb{I}_{A_t}\right]
    $$С другой стороны, по определению УМО:
    $$
        \E^{\PP} \left[\Lambda \cdot \mathbb{I}_{A_t}\right] = \E^{\PP} \left[\E^{\PP}\left[ \Lambda | \F_t \right] \cdot \mathbb{I}_{A_t}\right]
        = \E^{\PP} \left[\Lambda_t \cdot \mathbb{I}_{A_t}\right]
    $$А это и означает, что $\Lambda_t = \dfrac{d\Q_t}{d\PP_t}$
\end{frame}

\begin{frame}{Замена меры для процессов}
    \begin{block}{Теорема}
        Пусть $(\Omega, \F, \PP)$ -- вероятностное пространство, 
        $\Q \sim \PP$, $\Lambda = \dfrac{d\Q}{d\PP}$, $\Lambda_t = \E^{\PP}\left[ \Lambda | \F_t \right]$. 
        Тогда $X_t$ -- мартингал относительно $\Q$ тогда и только тогда, когда $\Lambda X_t$ -- мартингал относительно $\PP$. 
    \end{block}
    \textit{Доказательство}. Пусть $A_s \in \F_s$, тогда:
    $$
        \E^{\Q} \left[ X_t \mathbb{I}_{A_s}\right]
        = \E^{\PP} \left[ X_t \Lambda_t \mathbb{I}_{A_s}\right]
        = \E^{\PP} \left[ X_s \Lambda_s \mathbb{I}_{A_s}\right]
        = \E^{\Q} \left[ X_s\mathbb{I}_{A_s}\right]
    $$А это и означает, что $\E^{\Q} \left[ X_t | \F_s\right] = X_s$.
\end{frame}

\begin{frame}{Характеризация броуновского движения}
    \begin{block}{Теорема}
        Пусть $W_0 = 0$, $W_t, W_t^2 - t$ -- локальные мартингалы с непрерывными траекториями. Тогда $W_t$ -- броуновское движение.
    \end{block}
\end{frame}

\begin{frame}{Теорема Гирсанова: мотивация}
    Пусть $(\Omega, \F, \PP)$ -- вероятностное пространство, $(\F_t)_{0 \leq t \leq T}$ -- фильтрация.
    \begin{itemize}
        \item Пусть $(W_t)_{t\geq 0}$ -- броуновское движение. 
        \item Хотим найти меру $\Q$, относительно которой $Z_t = W_t - \theta \cdot t$ БД.
        \item Относительно $\Q$ $W_t \sim N(\theta t, t)$.
        \item "Кандидат" на производную Радона-Никодима:
        $$
            \Lambda = \exp(\theta \cdot W_T - 0.5 \theta^2 \cdot T)
        $$
        $$
            \Lambda_t = \E^{\PP} \left[ \Lambda | \F_t \right] = \exp(\theta \cdot W_t - 0.5 \theta^2 \cdot t)
        $$
\pause
        \item По формуле Ито:
        $$
            d\Lambda_t = \Lambda_t \theta dW_t
        $$
        \item Докажем, что $\Lambda_t Z_t$ мартингал относительно $\PP$:
        $$
            d(\Lambda_t Z_t) = 
            d\Lambda_t Z_t + \Lambda_t dZ_t + d\Lambda_t dZ_t
            = \Lambda_t \left( \theta Z_t dW_t + dW_t - \theta dt + \theta dt \right) =  
            \Lambda_t (1 + \theta Z_t ) dW_t
        $$
        \item Аналогично для $Z_t^2 - t$. Отсюда $Z_t$ -- броуновское движение относительно $\Q$.
    \end{itemize}
\end{frame}

\begin{frame}{Теорема Гирсанова}
    \begin{block}{Теорема}
        Пусть $\theta_t$ -- согласованный процесс, $\int_0^T \theta_s^2 ds < \infty$. Положим
        $$
            \Lambda = \exp(\int_0^T \theta_s dW_s - 0.5 \int_0^T \theta_s^2 ds)
        $$Определим новую меру $\Q = \Lambda \PP$, тогда процесс:
        $$
            Z_t = W_t - \int_0^t \theta_s ds
        $$является $\Q$-броуновским движением при $t \leq T$.
    \end{block}
\end{frame}

\begin{frame}{Риск-нейтральная мера}
    \begin{block}{Определение}
        Мера, относительно которой цена процесса подчиняются уравнению:
        $$
            dS_t = r S_t dt + \sigma S_t dW_t^Q
        $$ называется риск-нейтральной $\mathbb{Q}$. Здесь $W_t^Q$ -- $\Q$-броуновское движение.
    \end{block}
    \pause
    \begin{block}{Утверждение}
        Относительно риск-нейтральной меры дисконтированные цены $\widetilde{S_t} = \dfrac{S_t}{B_t}, \widetilde{V_t} = \dfrac{V_t}{B_t}$ мартингалы.
    \end{block}
    \pause
    \textit{Доказательство}. 
    \begin{align*}
        &d \widetilde S_t = \sigma \widetilde S_t dW_t \\
        &d \widetilde V_t = x_t d \widetilde S_t = \sigma x_t \widetilde S_t dW_t 
    \end{align*}
\end{frame}

\begin{frame}{Риск-нейтральная мера: существование}
    \begin{itemize}
        \item Динамика в реальной мере $\PP$:
        $$
            dS_t = \mu S_t dt + \sigma S_t dW_t
        $$
        \item Динамика в риск-нейтральной мере $\Q$:
        $$
            dS_t = r S_t dt + \sigma S_t dW_t^Q
        $$ 
        \item  Отсюда:
        $$
            W_t^Q = W_t + \lambda t 
        $$ где $\lambda = \dfrac{\mu - r}{\sigma}$ -- риск-премия.
        \item Производная Радона-Никодима:
        $$
            \Lambda_T = \dfrac{d\Q}{d\PP} = \exp(-\lambda \cdot W_T - 0.5 \lambda^2 \cdot T)
        $$
        \item В терминах нового БД:
        $$
            \Lambda_T = \dfrac{d\Q}{d\PP} = \exp(-\lambda \cdot W_T^Q + 0.5 \lambda^2 \cdot T)
        $$
    \end{itemize}
\end{frame}


\begin{frame}{Безарбитражность}
    Формула прайсинга:
    $$
        \dfrac{V_t}{B_t} = \mathbb{E}^{\mathbb{Q}}
        \left[ \dfrac{\Phi(S_T)}{B_T} \mid \mathcal{F}_t\right]
    $$
    %\pause 
    \begin{block}{Теорема}
        Модель Блэка-Шоулза безарбитражна. 
    \end{block}
    \textit{Доказательство}. Пусть $(h_t)_{t\geq 0}$ -- арбитражный портфель. Пусть $V_0^h = 0, V_T^h \geq 0$. Тогда:
    $$
        0 = V_0^h = \mathbb{E}^{\mathbb{Q}} \dfrac{V_T^h}{B_T} \to \mathbb{Q}(V_T = 0) = 1
        \to \PP(V_T=0)=1
    $$Пришли к противоречию.
\end{frame}

\begin{frame}{Полнота}
    \begin{block}{Определение}
        Платёжное обязательство $X\in\F_T$ называется реплицируемым, если $\exists$ самофинансируемый портфель $h_t=(x_t, y_t)$ такой, что:
        $$
            V_T^h \overset{a.s.}{=} X
        $$
    \end{block}
    \begin{block}{Определение}
        Рынок называется полный, если любое\textsuperscript{*} платёжное обязательство является реплицируемым.
    \end{block}
    \begin{itemize}
        \item На прошлой лекции доказали для случая $X = \Phi(S_T)$.
        \item Полнота эквивалентна представлению произвольного функционала от БД $X = X(\{W_s\}_{s \leq T})$ в виде интеграла Ито:
        $$
            X = \int_0^T g_s dW_s
        $$для некоторого адаптированного процесса $g_s$
    \end{itemize}
\end{frame}

\begin{frame}{Полнота: контр-пример}
    \begin{itemize}
        \item Пусть $W_t, Z_t$ -- два независимых БД, $\F_t = \sigma( \{W_s\}_{s\leq t}, \{Z_s\}_{s\leq t} )$.
        \item Динамика активов:
            \begin{align*}
                &dB_t = r B_t dt \\
                &dS_t = \sigma S_t dW_t
            \end{align*}
        \item Платёжное обязательство $X = \Phi(Z_T)$ не является реплицируемым $\to$ рынок не полный.
    \end{itemize}
    \begin{block}{Мета-теорема}
        Рынок полный $\Leftrightarrow$ число источинков случайности = числу рисковых активов.
    \end{block}
\end{frame}

\begin{frame}{Многомерная модель Блэка-Шоулза}
    \begin{itemize}
        \item $W_t^1, \ldots, W_t^K$ -- независимые броуновские движения
        \item $\F_t = \sigma(\{W_s^j\}_{s\leq t}, k=\overline{1,K})$ -- фильтрация
        \item Один безрисковый актив $S_t^0$ и $N$ рисковых активов:
        \begin{align*}
            &dS^0_t = r S^0_t dt \\
            &dS^i_t = S^i_t \left( \mu_i dt + \sum_{j=1}^K \sigma_{ij} dW_t^j \right)
        \end{align*}
        \item $\mu \in \R^N$ -- вектор доходностей
        \item $\sigma \in \R^{N\times K}, \sigma \sigma^{\top}$ -- матрица ковариаций лог-доходностей.
    \end{itemize}
\end{frame}

\begin{frame}{Динамика портфелей}
    \begin{itemize}
        \item Самофинансируемый портфель $h_t = (h_t^0, h_t^1, \ldots, h_t^N)$.
        \item Динамика самофинансируемого портфеля:
        $$
            dV_t^h = \sum_{i=0}^N h^t_i dS_t^i = h_t^0 dS^0_t + \sum_{i=1}^N h_t^i dS_t^i
        $$
        \item Динамика относительного портфеля $\tilde{V}^h_t = \dfrac{V_t^h}{B_t}$
        $$
            d\tilde{V}_t^h = \sum_{i=1}^N h_t^i d\tilde{S}_t^i
        $$
    \end{itemize}
\end{frame}

\begin{frame}{Риск-нейтральная мера}
    \begin{block}{Определение}
        Мера $\Q$ называется риск-нейтральной, если $\Q \sim \PP$ и рисковые активы имеют динамику:
        $$
            dS^i_t = S^i_t \left( r dt  + \sum_{j=1}^K \sigma_{ij} dZ_t^j \right)
        $$где $Z_t^j$ -- $\Q$-броуновские движения.
    \end{block}
\end{frame}

\begin{frame}{Риск-нейтральная мера}
    \begin{itemize}
        \item Замена дрифта у броуновского движения:
        $$
            W_t^j = Z_t^j - \lambda_j
        $$
        \item Дрифт относительно новой меры:
        $$
            dS^i_t = S^i_t \left( (\mu_i - \sum_{j=1}^K \sigma_{ij} \lambda_j) dt  + \sum_{j=1}^K \sigma_{ij} dZ_t^j \right)
        $$
        \item Уравнение на дрифт:
        $$
            \vec{\mu} - \sigma \vec{\lambda} = r \cdot \vec{1}
        $$
        \item В случае общего положения решение $\exists$, если $K \geq N$, решение $\exists!$, если $K=N$.
    \end{itemize}
\end{frame}

\begin{frame}{Первая фундаментальная теорема}
    \begin{block}{Первая фундаментальная теорема}
        Рынок безарбитражен $\Leftrightarrow$ существует риск-нейтральная мера.
    \end{block}

    $\Rightarrow$ Для случая $r=0$.
    \begin{itemize}
        \item От противного. Пусть система $\sigma \vec{\lambda} = \vec{\mu}$ не имеет решения.
        \item Альтернатива Фредгольма: система $\sigma^{\top} \vec{g} = \vec{0}$ имеет решение, $\langle \vec{g}, \mu \rangle \neq 0$.
        \item Портфель с весами $h_t^i = \frac{g_i}{S_t^i}, i \geq 1$. $h_t^0$ из условия самофинансируемости.
        \item Динамика портфеля:
        $$
            dV^h_t = \sum_{i=1}^n g_i \frac{dS_i}{S_i}
            = \langle \vec{g}, \vec{\mu} \rangle dt + 
            \langle \vec{g}, \hat{\sigma} d\vec{W}_t \rangle = 
            \langle \vec{g}, \vec{\mu} \rangle dt \neq 0
        $$ 
        \item Получили два безрисковых портфеля с разной доходностью $\to$ арбитраж.
    \end{itemize}
\end{frame}

\begin{frame}{Вторая фундаментальная теорема}
    \begin{block}{Вторая фундаментальная теорема}
        Рынок безарбитражный и полный $\Leftrightarrow$ риск-нейтральная мера единственна.
    \end{block}
\end{frame}

% \begin{frame}{Фундаментальные теоремы}
%     \begin{block}{Первая фундаментальная теорема}
%         Рынок безарбитражен $\Leftrightarrow$ существует риск-нейтральная мера.
%     \end{block}

%     \begin{block}{Вторая фундаментальная теорема}
%         Пусть рынок безарбитражен. Рынок полный $\Leftrightarrow$ риск-нейтральная мера единственна.
%     \end{block}
% \end{frame}

\begin{frame}{Эквивалентные мартингальные меры}
    \only<1->{\begin{block}{Определение}
        Пусть $N_t > 0$ -- numerair (единица измерения). Вероятностная мера $\Q_N$ называется 
        эквивалентной мартингальной мерой (EMM) относительно $N_t$, если 
        \begin{itemize}
            \item $\Q_N \sim \PP$
            \item Процессы $\tilde{B}_t = \dfrac{B_t}{N_t}, \tilde{S}_t = \dfrac{S_t}{N_t}$
            являются мартингалами относительно $\Q_N$
        \end{itemize}
    \end{block}}
    \only<1>{
    \begin{block}{Теорема}
        Пусть $\Q_N$ --- мартингальная мера, $V_t^h$ -- самофинансируемый портфель. Тогда 
        $$\tilde{V}_t^h = \dfrac{V_t^h}{N_t}$$ тоже самофинансируемый.          
    \end{block}}
    \only<2->{
    \begin{itemize}
        \item Стоимость $\tilde{V}_t$ любого самофинансируемого портфеля $\Q_N$-мартингал.
        \item Общая формула для ценообразования:
    $$
        \dfrac{V_t}{N_t} = \E^{Q_N} \left[ \dfrac{V_T}{N_T} | F_t \right]
    $$
    $$
        V_t = \E^{Q_N} \left[ \dfrac{N_t}{N_T} V_T | F_t \right]
    $$
    \end{itemize}}
\end{frame}
% \begin{frame}{Пример: Importance sampling}
%     \begin{itemize}
%         \item Пусть $\xi \sim N(0, 1)$.
%         \item Хотим оценить $p_a = \PP(\xi > a)$ при $a \gg 1$ с помощью метода Монте-Карло.
%         \item Выборка $\{\xi^{(j)}\}_{j=1}^{N}$, оцена 
%         $$
%             \hat{p}_a(N) = \dfrac{1}{N} \sum_{j=1}^N \mathbb{I}(\xi^{(j)} > a)
%         $$
%         \item Дисперсия:
%         $$
%             \mathrm{Var}(\hat{p}_a(N)) = \dfrac{\mathrm{Var}\mathbb{I}(\xi^{(j)} > a)}{N}
%             = \dfrac{p_a \cdot (1 - p_a)}{N}
%         $$
%         \item Относительная ошибка:
%         $$
%             \dfrac{\sqrt{\mathrm{Var}(\hat{p}_a(N))}}{\E (\hat{p}_a(N))} = \dfrac{\sqrt{1-p_a}}{\sqrt{N p_a}} \approx \dfrac{1}{\sqrt{N p_a}}
%         $$при $a \gg 1, p_a \ll 1$.
%     \end{itemize}
% \end{frame}

\begin{frame}{Пример}
    \begin{itemize}
        \only<1>{\item Контракт с пэйоффом $\Phi(S_T) = S_T \mathbb{I}(S_T \geq K)$
        \item Выберем $N_t = S_t$. Формула ценообразования: $$
        \dfrac{V_t}{S_t} = \E^{Q_S} \left[ \dfrac{V_T}{S_T} | \F_t \right]
        = \E^{Q_S} \left[ \mathbb{I}(S_T \geq K) | \F_t \right]
        = Q_S \left( S_T \geq K | \F_t \right)$$
        \item $Q_S$ -- EMM относительно $S_t$, $\tilde{S}_t = 1$ и 
        $\tilde{B_t}=\frac{B_t}{S_t}$ -- мартингалы.
        \item Процесс для $\tilde{B}_t$:
         $$
                \tilde{B}_t = \frac{1}{S_0}e^{rt} e^{-(r - 0.5 \sigma^2)t - \sigma W_t^Q}
                = \frac{1}{S_0}e^{0.5 \sigma^2 t - \sigma W_t^Q}
                = \frac{1}{S_0}e^{-0.5 \sigma^2 t - \sigma W_t^{Q_S}}
            $$ %$W_t^Q$ -- $\Q$-БД, $W_t^{Q_S}$ -- $\Q_S$-БД.
        }
        \only<1->{\item Динамика исходного актива:
            $$
                S_t = \frac{B_t}{\tilde{B}_t} = S_0 e^{(r + 0.5 \sigma^2) t + \sigma W_t^{Q_S}}
            $$}
        \only<2->{
            $$
                S_T \geq K \to (r + 0.5 \sigma^2) \tau + \sigma \xi \sqrt{\tau} \geq -\log \dfrac{S_t}{K}
                \to \xi \geq - \dfrac{\log (S_t / K) + (r + 0.5 \sigma^2 \tau)}{\sigma \sqrt{\tau}}
            $$
        \item Отсюда:
            $$
                V_t = S_t \cdot Q_S \left( S_T \geq K | \F_t \right) = S_t \cdot N(-d_1) = S_t \cdot N(d_1)
            $$}
    \end{itemize} 
\end{frame}
\end{document}
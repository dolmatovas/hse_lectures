\documentclass[12pt]{article}

\usepackage{preamble_problemset}
\usepackage{amsmath}

\begin{document}
\noindent Количественные финансы, осень 2025\hfill Семинар 4\\
\today

\hrulefill

\begin{problem}
    Пусть $C(t, S; T, K)$ -- цена колл-опциона в модели БШ:
    $$C(t, S; T, K) = S \cdot N(d_1) - e^{-r(T-t)} K \cdot N(d_2)$$
    где 
    \begin{align*}
        &d_1 =\dfrac{\log(S / K) + r(T-t)}{\sigma{\sqrt{T-t}}} + \dfrac{\sigma{\sqrt{T-t}}}{2}\\
        &d_2 = d_1 - \sigma \sqrt{T-t}
    \end{align*}
    Вычислить греки:
    \begin{align*}
        &\Delta = \dfrac{\partial C}{\partial S} \\
        &\Gamma = \dfrac{\partial^2 C}{\partial S^2} \\
        &\nu = \dfrac{\partial C}{\partial \sigma} \\
        &\Theta = \dfrac{\partial C}{\partial t}
    \end{align*}
\end{problem}\\
\noindent \textit{Решение.} Пусть $\tau = T - t$, $n(x) = N'(x) = \frac{\exp(-0.5x^2)}{\sqrt{2\pi}}$
$$
    \dfrac{\partial C}{\partial S} = 
    N(d_1) + S n(d_1) \dfrac{\partial d_1}{\partial S}
    - e^{-r\tau} K n(d_2) \dfrac{\partial d_2}{\partial S}
    = N(d_1) + \dfrac{\partial d_1}{\partial S}
    \left( S n(d_1) - e^{-r\tau} K n(d_2) \right)
$$
Рассмотрим выражение в скобках:
\begin{align*}
    n(d_2) = n(d_1 - \sigma \sqrt{\tau})
    = \frac{1}{\sqrt{2\pi}} \exp(-\frac{d_1^2 - 2d_1 \sigma \sqrt{\tau} + \sigma^2 \tau}{2})
= n(d_1) \exp(d_1 \sigma \sqrt{\tau} - \frac{\sigma^2 \tau}{2})
\end{align*}
Подставим определение $d_1$, получим:
$$
    d_1 \sigma \sqrt{\tau} - \frac{\sigma^2 \tau}{2}
    = \log(S / K) + r\tau
$$Откуда:
$$
    n(d_2) = n(d_1) e^{r\tau} \dfrac{S}{K}
$$Поэтому
$$
\left( S \cdot n(d_1) - e^{-r\tau} K \cdot n(d_2)\right) 
= n(d_1) \left( S - e^{-r\tau} K e^{r\tau} \dfrac{S}{K}\right) = 0
$$Итого
$$
\dfrac{\partial C}{\partial S} = N(d_1)
$$

$$
    \Gamma = \dfrac{\partial \Delta}{\partial S}
    = n(d_1) \dfrac{\partial d_1}{\partial S}
    = \dfrac{n(d_1)}{S \sigma \sqrt{\tau}}
$$

$$
    \dfrac{\partial C}{\partial \sigma}
    = S n(d_1) \dfrac{\partial d_1}{\parital \sigma}
    - e^{-r\tau} K n(d_2) \dfrac{\partial d_2}{\parital \sigma}
$$Заметим, что $\dfrac{\partial d_2}{\parital \sigma}
= \dfrac{\partial d_1}{\parital \sigma} - \sqrt{\tau}$.
Отсюда:
$$
    \dfrac{\partial C}{\partial \sigma}
    = e^{-r\tau} K n(d_2) \sqrt{\tau}
    + \dfrac{\partial d_1}{\parital \sigma} \left( Sn(d_1) - e^{-r\tau} K n(d_2)\right)
    = e^{-r\tau} K n(d_2) \sqrt{\tau}
$$Это также можно переписать как:
$$
    \dfrac{\partial C}{\partial \sigma}
    = n(d_1) S \sqrt{\tau}
$$
\begin{align*}
    &\dfrac{\partial C}{\partial t}
    = rC - \Delta r S - 0.5 \sigma^2 S^2 \Gamma =\\
    &= -rK e^{-r\tau} N(d_2) - \dfrac{\sigma S \phi(d_1)}{2 \sqrt{\tau}}
\end{align*}
\begin{problem}
    Рассмотрим контракт с пэйоффом:
    $$
        \Phi(S_T) = C(T, S_T; T', S_T)
    $$т.е. контракт, который в момент $T$ выплачивает стоимость ATM опциона с датой погашения $T'$.
    Найти его стоимость $p(t, \Phi)$.
\end{problem}\\
\noindent \textit{Решение}
$$
    \Phi(S_T) = S_T N(d_1) - e^{-r(T'-T)}S_TN(d_2)
    = S_T (N(d_1) - e^{-r(T'-T)} N(d_2))
$$где $d_1, d_2$ задаются формулами:
\begin{align*}
    &d_1 = \dfrac{\log(S_T / S_T) + r(T'-T)}{\sigma \sqrt{T'-T}}
    + \dfrac{\sigma \sqrt{T'-T}}{2}
    = \dfrac{r\sqrt{T'-T}}{\sigma} + \dfrac{\sigma \sqrt{T'-T}}{2}\\
    &d_2 = \dfrac{r\sqrt{T'-T}}{\sigma} - \dfrac{\sigma \sqrt{T'-T}}{2}
\end{align*}
Отсюда:
$$
    p(t, \Phi) = \E \left[ e^{-r(T-t)} \Phi(S_T) | F_t \right]
    = (N(d_1) - e^{-r(T'-T)} N(d_2)) \cdot 
    \E \left[ e^{-r(T-t)} S_T | F_t \right]
    = (N(d_1) - e^{-r(T'-T)} N(d_2)) \cdot S_t
$$
\end{document}

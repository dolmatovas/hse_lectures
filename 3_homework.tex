\documentclass[12pt]{article}

\usepackage{preamble_problemset}


\begin{document}
\noindent Количественные финансы, осень 2025\hfill Семинар 3\\
\today

\hrulefill
\begin{problem}
    Пусть
    $$\begin{cases}
        dX_t = X_t (\mu_x dt + \sigma_x dB_t), \\
        dY_t = Y_t (\mu_y dt + \sigma_y dZ_t),
    \end{cases}$$
    где $dB_t\cdot dZ_t = \rho dt$ -- броуновские движения с корреляций $\rho$, $\mu_x, \mu_y, \sigma_x, \sigma_y$ -- константы.
    \\ Выписать уравнения для процессов $X_t^{\alpha}, X_t \cdot Y_t, \dfrac{X_t}{Y_t}, \alpha \in \R$.
\end{problem}

% \begin{problem}[Variance swap]
%     Пусть $dX_t = X_t \sigma_t dB_t$ -- процесс Ито, $\sigma_t$ -- согласованный процесс.
%     \\ Покажите, что:
%     $$
%         \int_0^T \sigma^2_t dt = -2\ln \dfrac{X_T}{X_0} + \int_0^T \dfrac{2}{X_t}dX_t
%     $$
% \end{problem}

\begin{problem}[Броуновский мост]
    Пусть $X_t$ удовлетворяет СДУ:    
    $$
        dX_t = a(t) X_t + dB_t
    $$ где $a(t)$ -- детерменированная функция, $B_t$ -- броуновское движение. 
    Найдите $a(t)$ такое, что процесс $X_t$, определённый по формуле выше, является броуновским мостом.
    \\ Броуновский мост это гауссовский процесс $X_t$: $\E X_t = 0, \; \mathrm{cov}(X_t, X_s) = s\cdot (1 - t), \; s \leq t$
\end{problem}

\begin{problem}[Формула Феймана-Каца]
    Пусть $f$ удовлетворяет УРЧП
    \begin{align*}
        &f_t + \mu(t, x) f_x + \dfrac{\sigma^2(t, x)}{2} f_{xx} = rf, 0 \leq t < T \\
        &f(T, x) = \Phi(x)
    \end{align*}
    где $r \in \R$. Докажите, что:
    $$
        f(t, x) = \E \left[ e^{-r(T-t)} \Phi(X_T) | X_t = x\right]
    $$
\end{problem}

\begin{problem}[Процесс Орнштейна-Уленбека]
    Пусть $X_t$ удовлетворяет СДУ:
    \begin{align*}
        &dX_t = \alpha (\theta - X_t) dt + \sigma dW_t \\
        &X_0 = x_0
    \end{align*}
    Выпишите прямое уравнение Колмогорова на плотность процесса $X_t$. Найдите стационарное решение (плотность, для которой $\dfrac{\partial p(t, x)}{\partial t} = 0$).
\end{problem}

\begin{problem}
    Пусть $u(x, y)$ удовлетворяет уравнению Лапласа в области $x^2+y^2 \leq 1$:
    $$
        \dfrac{\partial^2 u}{\partial x^2} + \dfrac{\partial^2 u}{\partial y^2} = 0 
    $$и граничным условиям $u(x, y) = f(x, y)$ при $x^2+y^2=1$. 
    \\ Доказать, что:
    $$
        u(x, y) = \E \left[f(X_{\tau}, Y_{\tau}) | (X_0 = x, Y_0 = y)\right]
    $$где $(X_t, Y_t)$ -- двумерное броуновское движение, стартующее из точки $(x, y)$, момент остановки $\tau$ определяется как:
    $$
        \tau = \inf_{t} \{ X_t^2 + Y_t^2 \geq 1 \}
    $$
\end{problem}
\end{document}
